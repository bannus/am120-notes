\documentclass[12pt]{article}
\usepackage[margin=1.25in]{geometry}
\usepackage{amsmath}
\usepackage{amsthm}
\usepackage{hyperref}
\newtheorem{mydef}{Definition}

%%%%%%%%%%%%%
% These are a few commands which I have found useful over the years..
%%%%%%%%%%%%%
\newcommand{\norm}[1]{\left|\left|#1\right|\right|}
\newcommand{\abs}[1]{\left|#1\right|}
\newcommand{\bx}{\mathbf{x}}
\newcommand{\bu}{\mathbf{u}}
\newcommand{\floor}[1]{\lfloor #1 \rfloor}
\newcommand{\ceil}[1]{\lceil#1 \rceil}
\newcommand{\bv}{\mathbf{v}}
\newcommand{\bbZ}{\mathds{Z}}
\newcommand{\bbR}{\mathds{R}}
\newcommand{\bbC}{\mathds{C}}
\newcommand{\bbN}{\mathds{N}}
\newcommand{\bbQ}{\mathds{Q}}
\newcommand{\bbzw}{\mathds{Z}[\omega]}
\newcommand{\bbzi}{\mathds{Z}[i]}
\newcommand{\nZ}[1][n]{\ensuremath{\mathds{Z}/#1\mathds{Z}}}
\newcommand{\<}{\langle}
\renewcommand{\>}{\rangle}

\title{AM 120 Notes}
\date{\today}
\author{Bannus Van der Kloot}

\begin{document}
\maketitle

\tableofcontents

\section{September 6 Lecture}

There are two main problems that we will learn how to handle in this class.

\begin{enumerate}
	\item Find $x \in R^n$ such that $Ax=b$. $A$ is $m$ by $n$ matrix, $b \in R^n$ vector
	\item Find $x$ and $\lambda$ such that $Ax = \lambda x$
\end{enumerate}

\paragraph{Example}

\begin{tabular}{rr}
	$x + 2y = 3$ \\
	$4x + 5y = 6$
\end{tabular}

$$ A =
\begin{bmatrix}
	1 & 2 \\
	4 & 5 
\end{bmatrix}
\begin{bmatrix}
	x \\ y
\end{bmatrix}
=
\begin{bmatrix}
	3 \\ 6
\end{bmatrix}
$$

There are 3 ways to solve:
\begin{enumerate}
	\item $\begin{bmatrix}
		1 & 2 & 3 \\
		4 & 5 & 6
	\end{bmatrix} \Rightarrow
	\begin{bmatrix}
		1 & 2 & 3 \\
		0 & -3 & -6
	\end{bmatrix}\Rightarrow
	y=2,x=-1$ 
	\item $A^{-1}=\frac{1}{\text{det}(A)}
	\begin{bmatrix}
		5 & -2 \\
		-4 & -1 
	\end{bmatrix}$
	$det A = -3$ \\
	$x=A^{-1}b \Rightarrow \frac{1}{-3}
	\begin{bmatrix}
		+3 \\ -6
	\end{bmatrix}$
	\item Kramer's rule
\end{enumerate}

\paragraph{Summary}
Topics covered in next 3 classes:
\begin{enumerate}
	\item Geometric interpretation of solving linear systems
	\item Matrix notation (LU factorization)
	\item Singular cases (no solution, multiple soln's)
	\item Efficient way to solve $Ax=b$ using computers
\end{enumerate}

\paragraph{Example} Graphical method:

Row interpretation (plot lines on coordinate system):

\begin{tabular}{c}
	$2x - y = 1$ \\
	$x + y = 5$
\end{tabular}
Solution: $x=2,y=3$

Column interpretation:

$$ \begin{bmatrix}
	2 \\ 1
\end{bmatrix} x + 
\begin{bmatrix}
	-1 \\ 1
\end{bmatrix} y =
\begin{bmatrix}
	1 \\ 5
\end{bmatrix}$$

\paragraph{Example} 3 by 3 system:

Each row represents a plane:

\begin{tabular}{c}
	2u + v + w = 5 \\
	4u - 6v + 0 = -2 \\
	-2u + 7v + 2w = 9
\end{tabular}

Remember: inner product of vector with another vector equals 0 $\Rightarrow$ orthogonal.

Column interpretation:

\[
	\begin{bmatrix}
		2 \\ 4 \\ 2
	\end{bmatrix} u + 
	\begin{bmatrix}
		1 \\ -6 \\ 7
	\end{bmatrix} v + 
	\begin{bmatrix}
		1 \\ 0 \\ 2
	\end{bmatrix} w =
	\begin{bmatrix}
		5 \\ -2 \\ 9
	\end{bmatrix}
\]

\paragraph{Example} Overdetermined system:
\[
	\begin{bmatrix}
		1 & 1 \\
		2 & 3 \\
		3 & 4 \\
	\end{bmatrix}
	\begin{bmatrix}
		c \\ d
	\end{bmatrix} = 
	\begin{bmatrix}
		2 \\ 5 \\ 7
	\end{bmatrix}
\]

Solution: $c = 1, d = 1$

In 4 dimensions, the rows represent 3-spaces, which are `flat' relative to 4 dimensional space. If we intersect $(x,y,z,t=0)$ with $(x,y,z=0,t)$, two three spaces, we get $(x,y)$ plane.

\[
	a_1 u + a_2 v + a_3 w + a_4 z = b
\]

\[
	A = (a_1 | a_2 | a_3 | a_4)
\]

\subsection{Algorithmic approach}
Generalizing to $n$ by $n$. How to solve $Ax = b$ in a way that scales well? Gaussian elimination (row reduction).


\begin{tabular}{c}
	$2u + v + w = 5$ \\
	$4u - 6v + 0 = -2$ \\
	$-2u + 7v + 2w = 9$ \\\hline
\end{tabular}

$\Rightarrow$
\begin{tabular}{c}
	$2u + v + w = 5$ \\
	   $-8v - 2w = -12$ \\
         $8v + 3w = 14$ \\\hline
\end{tabular}

$\Rightarrow$
\begin{tabular}{c}
	$2u + v + w = 5$ \\
	   $-8v - 2w = -12$ \\
         $w = 2$ \\\hline
\end{tabular}

$\Rightarrow v = 1, u = 1$

We need a process that takes:

\[
	A = \begin{bmatrix}
		2 & 1 & 1 \\
		4 & -6 & 0 \\
		-2 & 7 & 2 \\
	\end{bmatrix}
\]
\[
	x = \begin{bmatrix}
		u \\ v \\ w
	\end{bmatrix}
\]
\[
	b = \begin{bmatrix}
		5 \\ -2 \\ 9
	\end{bmatrix}
\]

...this $Ax=b$ problem and transforms it to a $Ux = \hat{b}$ problem. We can get an upper triangular matrix, and obtain solution by back substitution.

\paragraph{Problems} One issue that could arise is if the bottom row is all 0s: infinitely many solutions.

\section{September 11 Lecture}

Last class:
\begin{itemize}
	\item Introduced first central problem of linear algebra: solving linear equations
	\item Studied column and row interpretation of linear systems
	\item Introduced Gaussian elimination
\end{itemize}

\paragraph{Example} (from previous class) Row/Column interpretation.

\begin{tabular}{c}
	2u + v + w = 5 \\
	4u - 6v + 0 = -2 \\
	-2u + 7v + 2w = 9
\end{tabular}

Row: Three planes intersecting. Column: linear combination of three vectors

\[
	A = \begin{bmatrix}
		2 & 1 & 1 \\
		4 & -6 & 0 \\
		-2 & 7 & 2 \\
	\end{bmatrix}
\]

We were trying to figure out how to transform matrix $A$ into an upper triangular matrix.

\[
	\begin{bmatrix}
		1 & 0 & 0 \\
		-2 & 1 & 0 \\
		0 & 0 & 1 \\
	\end{bmatrix}
	\begin{bmatrix}
		5 \\ -2 \\ 9
	\end{bmatrix}
	=
	\begin{bmatrix}
		5 \\ -12 \\ 9
	\end{bmatrix}
\]

\[
	\begin{bmatrix}
		1 & 0 & 0 \\
		-2 & 1 & 0 \\
		0 & 0 & 1 \\
	\end{bmatrix}
	\begin{bmatrix}
		2 & 1 & 1 \\
		4 & -6 & 0 \\
		-2 & 7 & 2 \\
	\end{bmatrix} = 
	\begin{bmatrix}
		2 & 1 & 1 \\
		0 & -8 & -2 \\
		-2 & 7 & 2 \\
	\end{bmatrix}
\]

\paragraph{Matrix operations} Addition is associative: $A+B+C = (A+B) + C = A+ (B+C)$

Multiplication: dimension $m \times n$ multiplied by $n \times p$ results in $m \times p$ matrix. $AB \neq BA$.

\[
	\begin{bmatrix}
		0 & 1 \\ 1 & 0\\
	\end{bmatrix}
	\begin{bmatrix}
		2 & 3 \\ 7 & 8 \\ 
	\end{bmatrix} =
	\begin{bmatrix}
		7 & 8 \\ 2 & 3
	\end{bmatrix}
\]

\[
	\begin{bmatrix}
		2 & 3 \\ 7 & 8 \\ 
	\end{bmatrix} 
	\begin{bmatrix}
		0 & 1 \\ 1 & 0\\
	\end{bmatrix}=
	\begin{bmatrix}
		3 & 2 \\ 8 & 7
	\end{bmatrix}
\]

Matrix multiplication:
\[
	\begin{bmatrix}
	a_{11} & a_{12} & a_{13} & ... & a_{1n} \\
	a_{21} & a_{22} & a_{23} & ... & a_{2n} \\
	... & &  & ... &  \\
	a_{m1} & a_{m2} & a_{m3}& ... & a_{mn} \\
	\end{bmatrix}
	\begin{bmatrix}
		x_1 \\ x_2 \\ ... \\ x_n
	\end{bmatrix} = 
	\begin{bmatrix}
		\sum_{i=1}^n a_{1i}x_i \\
		\sum_{i=1}^n a_{2i}x_i \\ 
		... \\
		\sum_{i=1}^n a_{ni}x_i\\
	\end{bmatrix}
\]

\[
	\begin{bmatrix}
		| & | &  & | \\
		a_1 & a_2 & ... & a_n \\
		| & | &  & | \\
	\end{bmatrix}
		\begin{bmatrix}
		x_1 \\ ... \\ x_n
	\end{bmatrix} = 
	a_1x_1+a_2x_2+...+a_nx_n
\]

\[
	\begin{bmatrix}
		| & | &  & | \\
		a_1 & a_2 & ... & a_n \\
		| & | &  & | \\
	\end{bmatrix}
	\begin{bmatrix}
		| &| \\
		b_1 & b_2 \\
		| & | \\
	\end{bmatrix} = 
	\begin{bmatrix}
		| & | \\
		Ab_1 & Ab_2 \\
		| & | \\
	\end{bmatrix} 
\]

\paragraph{Row reduction} In matrix form

\[
	A = \begin{bmatrix}
		2 & 1 & 1 \\
		4 & -6 & 0 \\
		-2 & 7 & 2 \\
	\end{bmatrix}
\]

\begin{enumerate}
	\item Subtract 2 times row 1 to row 2
	\[
		\begin{bmatrix}
			1 & 0 & 0 \\
			-2 & 1 & 0 \\
			0 & 0 & 1 \\
		\end{bmatrix}_{E_{21}}
		\begin{bmatrix}
			2 & 1 & 1 \\
			4 & -6 & 0 \\
			-2 & 7 & 2 \\
		\end{bmatrix}_A = 
		\begin{bmatrix}
			2 & 1 & 1 \\
			0 & -8 & -2 \\
			-2 & 7 & 2 \\
		\end{bmatrix}
	\]
	\item Subtract -1 times row 1 to row 3
	\[
		\begin{bmatrix}
			1 & 0 & 0 \\
			0 & 1 & 0 \\
			1 & 0 & 1 \\
		\end{bmatrix}_{E_{31}}
		E_{21}A = 
		\begin{bmatrix}
			2 & 1 & 1 \\
			0 & -8 & -2 \\
			0 & 8 & 3 \\
		\end{bmatrix}
	\]
	\item Subtract -1 times row 2 to row 3
	\[
		\begin{bmatrix}
			1 & 0 & 0 \\
			0 & 1 & 0 \\
			0 & 1 & 1 \\
		\end{bmatrix}_{E_{32}}
		E_{31}E_{21}A = 
		\begin{bmatrix}
			2 & 1 & 1 \\
			0 & -8 & -2 \\
			0 & 0 & 1 \\
		\end{bmatrix}
	\]
\end{enumerate}

Originally we wanted to solve $Ax=b$. Now we have:

\[
	E_{32}E_{31}E_{21}A = U
\]

where $U$ is an upper triangular matrix.

\[
	E_{32}E_{31}E_{21}Ax = Ux
\]

Let's let $E_{32}E_{31}E_{21} = L$. Then, we have

\[
	\begin{matrix}
		L^{-1}A=U \\
		A = LU \\
		Ux = C = E_{32}E_{31}E_{21}b
	\end{matrix}
\]

Now we can solve by back substitution.

\[
	L^{-1}=E_{32}E_{31}E_{21} = \begin{bmatrix}
		1 & 0 & 0 \\
		-2 & 1 & 0 \\
		-1 & 1 & 1 
	\end{bmatrix}
\]


Matrix inverse properties:

\[
	\begin{matrix}
		(AB)^{-1} = B^{-1}A^{-1} \\
		(A_1A_2...A_n)^{-1} = A_n^{-1}...A_2^{-1}A_1^{-1}
	\end{matrix}
\]

So we have:

\[
	L = \begin{bmatrix}
		1 & 0 & 0 \\
		2 & 1 & 0 \\
		-1 & -1 & 1
	\end{bmatrix} = E_{21}^{-1}E_{31}^{-1}E_{32}^{-1}
\]

\paragraph{Row reduction matrices} A matrix that subtracts $l$ times row $j$ from row $i$ is such that it includes $-l$ in row $i$, column $j$.

\[
	\begin{matrix}
			A = \begin{bmatrix}
		1 & 0 & 0 \\
		2 & 1 & 0 \\
		-1 & -1 & 1
	\end{bmatrix}_L
	\begin{bmatrix}
		2 & 1 & 1 \\
		0 & -8 & -2 \\
		0 & 0 & 1 \\
	\end{bmatrix}_U
	\end{matrix}
\]

$L$ is lower triangular and $U$ is upper triangular.

\begin{enumerate}
	\item Compute LU factorization
	\item Solve for $c$ in $Lc = b$ (forward substitution)
	\item Solve for $x$ in $Ux = c$ (back substitution)
\end{enumerate}

We want to solve $Ax=b$. We factor to get $LUx =b$. First we find $c$ such that $Lc = b$

\subsection{General Example}
\[
	\begin{bmatrix}
		l_{11} & 0 & 0 \\
		l_{21} & l_{22} & 0 \\
		l_{31} & l_{32} & l_{33}
	\end{bmatrix}
	\begin{bmatrix}
		c_1 \\ c_2 \\ c_3
	\end{bmatrix} = 
	\begin{bmatrix}
		b_1 \\ b_2 \\ b_3
	\end{bmatrix} \Rightarrow
	\begin{matrix}
		c_1 = b_1 / l_{11} \\
		c_2 = b_2 - b_1 l_{21}/ l_{11} \\
		c_3 = b_3 - l_{31}b_1 - l_{32}(b_2 - b_1 l_{21})
	\end{matrix}
\]

\[
	\begin{bmatrix}
		u_{11} & u_{12} & u_{13} \\
		0 & u_{22} & u_{23} \\
		0 & 0 & u_{33}
	\end{bmatrix}
	\begin{bmatrix}
		x_1 \\ x_2 \\ x_3
	\end{bmatrix} = 
	\begin{bmatrix}
		c_1 \\ c_2 \\ c_3
	\end{bmatrix} \Rightarrow
	\begin{matrix}
		x_3 = c_3/u_{33} \\
		x_2 = \frac{1}{u_{22}}(c_2 - u_{23}c_3/u_{33})\\
		x_1 = .....
	\end{matrix}
\]

\section{September 13 Lecture}

Announcements

\begin{itemize}
	\item Matlab tutorials (sections)
	\item Final projects
	\begin{itemize}
		\item Adjustment based on class size
		\item Pairs
	\end{itemize}
	\item Assignment 1 due Fri @ 7pm in Pierce 303
	\item Collaboration policy
\end{itemize}

From last time:

\begin{itemize}
	\item Linear equations $\rightarrow$ Matrix notation
	\item Column $j$ of $AB=Ab_j$
	\[
		A
		\begin{bmatrix}
			| & | &  & | \\
			b_1 & b_2 & ... & b_n \\
			| & | &  & | \\
		\end{bmatrix} = 
		\begin{bmatrix}
			Ab_1 & Ab_2 & ... & Ab_n \\
		\end{bmatrix} 
	\]
	\item Introduced the $LU$ factorization of square matrix $A$ (see general example at end of last lecture)
	\[
		Ax=b \Rightarrow LUx = b
	\]
\end{itemize}

\begin{enumerate}
	\item Find $LU$
	\item Solve for $c$ in $Lc=b$
	\item Solve for $x$ in $Ux=c$
\end{enumerate}

\paragraph{Example} $LU$ factorization

\[
	A = 
	\begin{bmatrix}
		1 & 0 & 1 \\ 
		2 & 2 & 2 \\
		3 & 4 & 5
	\end{bmatrix}
\]

\begin{enumerate}
	\item Subtract 2 times row 1 to row 2
	\[
		\begin{bmatrix}
			1 & 0 & 0 \\ 
			-2 & 1 & 0 \\
			0 & 0 & 1
		\end{bmatrix}_{E_{21}} A =
		\begin{bmatrix}
			1 & 0 & 1 \\ 
			0 & 2 & 0 \\
			3 & 4 & 5
		\end{bmatrix}
	\]
	\[
		E_{21}^{-1} = \begin{bmatrix}
			1 & 0 & 0 \\ 
			2 & 1 & 0 \\
			0 & 0 & 1
		\end{bmatrix}
	\]
	\item Subtract 3 times row 1 to row 2
	\[
		\begin{bmatrix}
			1 & 0 & 0 \\ 
			0 & 1 & 0 \\
			-3 & 0 & 1
		\end{bmatrix}_{E_{31}} E_{21} A =
		\begin{bmatrix}
			1 & 0 & 1 \\ 
			0 & 2 & 0 \\
			0 & 4 & 2
		\end{bmatrix}
	\]
	\item Subtract 2 times row 2 to row 3
	\[
		\begin{bmatrix}
			1 & 0 & 0 \\ 
			0 & 1 & 0 \\
			0 & -2 & 1
		\end{bmatrix}_E{32} E_{31} E_{21} A =
		\begin{bmatrix}
			1 & 0 & 1 \\ 
			0 & 2 & 0 \\
			0 & 0 & 2
		\end{bmatrix}_U
	\]
\end{enumerate}

\[
	L^{-1} = E_{32} E_{31} E_{21} = \begin{bmatrix}
		1 & 0 & 0
		-2 & 1 & 0
		-3 & -2 & 1
	\end{bmatrix}
\]

\[
	\begin{matrix}
		L^{-1}A = U \\
		L = E_{21}^{-1} E_{31}^{-1} E_{32}^{-1} \\
		L = \begin{bmatrix}
			1 & 0 & 0 \\
			2 & 1 & 0 \\
			3 & 2 & 0 \\
		\end{bmatrix}
	\end{matrix}
\]

\paragraph{Generalizing $LU$ factorization} To $n \times n$ matrix:
\[
	\begin{bmatrix}
		a_{11} & a_{12} & a_{13} & ... &  a_{1n} \\
		a_{21} & a_{22} & a_{23} & ... &  a_{2n} \\
		... & ... & ... & ... & ...  \\
		a_{n1} & a_{n2} & a_{n3} & ... &  a_{nn} \\
	\end{bmatrix}
\]

\begin{enumerate}
	\item Introduce zeros below $a_{11}$ by subtracting multiples of row 1
	\item Use multipliers $l = \frac{a_i1}{a_11}$
	\item Repeat 1 and 2 for $a_{22}^*,a_{33}^*$, ...
\end{enumerate}

Step 1:
\[
	\begin{array}{c|cccc}
		a_{11} & a_{12} & a_{13} & ... &  a_{1n} \\\hline
		0 & a_{22}^* & a_{23}^* & ... &  a_{2n}^* \\
		... & ... & ... & ... & ...  \\
		0 & a_{n2}^* & a_{n3}^* & ... &  a_{nn}^* \\
	\end{array}
\]

Step 2:
\[
	\begin{array}{c|c|ccc}
		a_{11} & a_{12} & a_{13} & ... &  a_{1n} \\\hline
		0 & a_{22}^* & a_{23}^* & ... &  a_{2n}^* \\\hline
		... & 0 & a_{33}^* & ... & a_{3n}^* \\
		... & ... & ... & ... & ...  \\
		0 & 0 & a_{n3}^* & ... &  a_{nn}^* \\
	\end{array}
\]

How many operations does this algorithm use?

\[
	\sum_{k=1}^n k^2 - \sum_{k=1}^n k = \frac{n(n+1)(2n+1)}{6} - \frac{n(n+1)}{2}
\]

\section{September 18 Lecture}

To review: solving $Ax=b$:
\begin{enumerate}
	\item Find $LU=A$
	\item Solve for $c$ in $Lc=b$ (forward substitution)
	\item Solve for $x$ in $Ux=c$ (back-subst)
\end{enumerate}

Multipliers to find $U$ are entries of $L$.

What is the \# of operations needed to get $LU$ factorization?

\[
	\approx {n^3 - n \over 3}
\]

\paragraph{Forward substitution} Number of operations:

$(n-1) + (n-2) + ... (1) \approx O(n^2)$

Back substitution is similar process (also $O(n^2)$). Most time consuming place is step 1.

\paragraph{Algorithm Failure} This $Ax=b$:

\[
	\begin{bmatrix}
		0 & 1 \\ 1 & 0
	\end{bmatrix}
	\begin{bmatrix}
		x_1 \\ x_2
	\end{bmatrix} = 
	\begin{bmatrix}
		1 \\ 1
	\end{bmatrix}
\]

has solution $\begin{bmatrix}
	1 \\ 1
\end{bmatrix}$. However, our algorithm won't find the answer because it can't switch rows. If the algorithm fails we have two options:

\begin{enumerate}
	\item We need to rearrange rows
	\item No solution
	\item Infinitely many solutions
\end{enumerate}

Example of (2):

\[
	\begin{bmatrix}
		0 & 1 \\ 0 & 0
	\end{bmatrix}
	\begin{bmatrix}
		x_1 \\ x_2
	\end{bmatrix} = 
	\begin{bmatrix}
		1 \\ 1
	\end{bmatrix}
\]

Example of (3):

\[
	\begin{bmatrix}
		0 & 1 \\ 0 & 1
	\end{bmatrix}
	\begin{bmatrix}
		x_1 \\ x_2
	\end{bmatrix} = 
	\begin{bmatrix}
		1 \\ 1
	\end{bmatrix}
\]

\paragraph{Fact} $\det(A) = \det(LU) = \det(L)\det(U)$

\[
	\det(U) = \prod_{i=1}^n u_{ii}
\]

\paragraph{Example} Consider this:

\[
	\begin{bmatrix}
		0.0001 & 1 \\ 1 & 1
	\end{bmatrix}
	\begin{bmatrix}
		x_1 \\ x_2
	\end{bmatrix} = 
	\begin{bmatrix}
		1 \\ 2
	\end{bmatrix}
\]

\[
	\begin{bmatrix}
		0.0001 & 1 \\ 0 & -9999
	\end{bmatrix}
	\begin{bmatrix}
		x_1 \\ x_2
	\end{bmatrix} = 
	\begin{bmatrix}
		1 \\ -9998
	\end{bmatrix}
\]

\[
	\Rightarrow x_2 = {9998 \over 9999}
\]

\[
	0.0001 x_1 + {9998 \over 9999} = 1
\]

\[
	\Rightarrow x_1= {10000 \over 9999}
\]

If we do all of this with limited precision (say 3 digits), we do the following:

\[
	\begin{bmatrix}
		0.0001 & 1 \\ 0 & -10^4
	\end{bmatrix}
	\begin{bmatrix}
		x_1 \\ x_2
	\end{bmatrix} = 
	\begin{bmatrix}
		1 \\ -10^4
	\end{bmatrix}
\]

\[
	\Rightarrow x_2 = 1
\]

Then if we use the first equation, we get

\[
	\Rightarrow x_1 = 0
\]

This is called \textbf{catastrophic cancellation}.

\section{September 20 Lecture}
First part of AM120 is to solve $Ax=b$ for arbitrary $\norm{A} = n$.

\[
	u_{11} = a_{11}, u_{22} = a_{22} 
\]

Pseudocode did not have 0s in $L$ and $U$. Second part of code is given $L$ and $b$, should output $c$. Third part takes $U$ and $c$ and outputs $x$.

Assignment 2 Due on Monday morning (9am).

This Doolittle algorithm can fail:
\begin{enumerate}
	\item If there is a pivot = 0
	\begin{enumerate}
		\item System is singular $\Rightarrow \det(A) = 0$. This means there is no solution or infinitely many solutions.
		\item We can exchange rows and `cure' system.
		\[
			\det(A) = \det(L) \det(U) = 1 \prod_{k=1}^n u_{kk}
		\]
	\end{enumerate}
\end{enumerate}


\paragraph{Example} From last class:

\[
	\begin{bmatrix}
		0.0001 & 1 \\ 1 & 1
	\end{bmatrix}
	\begin{bmatrix}
		x_1 \\ x_2
	\end{bmatrix} = 
	\begin{bmatrix}
		1 \\ 2
	\end{bmatrix}
\]

This had true solution:

\[
		x_1 = {10000 \over 9999}, x_2 = {9998 \over 9999}
\]

But with limited precision (three digit arithmetic), we got:

\[
		x_1 = 0, x_2 = 1
\]

What if we switch the rows?

\[
	\begin{bmatrix}
		1 & 1 \\ 0.0001 & 1
	\end{bmatrix}
	\begin{bmatrix}
		x_1 \\ x_2
	\end{bmatrix} = 
	\begin{bmatrix}
		2 \\ 1
	\end{bmatrix}
\]

\[
	\begin{bmatrix}
		1 & 0 \\ 10^{-4} & 1
	\end{bmatrix}_L
	\begin{bmatrix}
		1 & 1 \\ 0 & 1
	\end{bmatrix}
	\begin{bmatrix}
		x_1 \\ x_2
	\end{bmatrix} =
	\begin{bmatrix}
		2 \\ 1 - 2 \cdot 10^{-4}t
	\end{bmatrix}
\]



\end{document}