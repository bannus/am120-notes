\documentclass[12pt]{article}
\usepackage[margin=1.25in]{geometry}
\usepackage{amsmath}
\usepackage{amsthm}
\usepackage{hyperref}
\usepackage{dsfont}
\newtheorem{mydef}{Definition}

%%%%%%%%%%%%%
% These are a few commands which I have found useful over the years..
%%%%%%%%%%%%%
\newcommand{\norm}[1]{\left|\left|#1\right|\right|}
\newcommand{\abs}[1]{\left|#1\right|}
\newcommand{\bx}{\mathbf{x}}
\newcommand{\bu}{\mathbf{u}}
\newcommand{\floor}[1]{\lfloor #1 \rfloor}
\newcommand{\ceil}[1]{\lceil#1 \rceil}
\newcommand{\bv}{\mathbf{v}}
\newcommand{\bbZ}{\mathds{Z}}
\newcommand{\bbR}{\mathds{R}}
\newcommand{\bbC}{\mathds{C}}
\newcommand{\bbN}{\mathds{N}}
\newcommand{\bbQ}{\mathds{Q}}
\newcommand{\bbzw}{\mathds{Z}[\omega]}
\newcommand{\bbzi}{\mathds{Z}[i]}
\newcommand{\nZ}[1][n]{\ensuremath{\mathds{Z}/#1\mathds{Z}}}
\newcommand{\<}{\langle}
\renewcommand{\>}{\rangle}

\title{AM 120 Notes}
\date{\today}
\author{Bannus Van der Kloot}

\begin{document}
\maketitle

\tableofcontents

%!TEX root = ../notes.tex
\section{September 6 Lecture}

There are two main problems that we will learn how to handle in this class.

\begin{enumerate}
	\item Find $x \in R^n$ such that $Ax=b$. $A$ is $m$ by $n$ matrix, $b \in R^n$ vector
	\item Find $x$ and $\lambda$ such that $Ax = \lambda x$
\end{enumerate}

\paragraph{Example}

\begin{tabular}{rr}
	$x + 2y = 3$ \\
	$4x + 5y = 6$
\end{tabular}

$$ A =
\begin{bmatrix}
	1 & 2 \\
	4 & 5 
\end{bmatrix}
\begin{bmatrix}
	x \\ y
\end{bmatrix}
=
\begin{bmatrix}
	3 \\ 6
\end{bmatrix}
$$

There are 3 ways to solve:
\begin{enumerate}
	\item $\begin{bmatrix}
		1 & 2 & 3 \\
		4 & 5 & 6
	\end{bmatrix} \Rightarrow
	\begin{bmatrix}
		1 & 2 & 3 \\
		0 & -3 & -6
	\end{bmatrix}\Rightarrow
	y=2,x=-1$ 
	\item $A^{-1}=\frac{1}{\text{det}(A)}
	\begin{bmatrix}
		5 & -2 \\
		-4 & -1 
	\end{bmatrix}$
	$det A = -3$ \\
	$x=A^{-1}b \Rightarrow \frac{1}{-3}
	\begin{bmatrix}
		+3 \\ -6
	\end{bmatrix}$
	\item Kramer's rule
\end{enumerate}

\paragraph{Summary}
Topics covered in next 3 classes:
\begin{enumerate}
	\item Geometric interpretation of solving linear systems
	\item Matrix notation (LU factorization)
	\item Singular cases (no solution, multiple soln's)
	\item Efficient way to solve $Ax=b$ using computers
\end{enumerate}

\subsection{Geometric interpretation}

\paragraph{Example} Graphical method:

Row interpretation (plot lines on coordinate system):

\begin{tabular}{c}
	$2x - y = 1$ \\
	$x + y = 5$
\end{tabular}
Solution: $x=2,y=3$

Column interpretation:

$$ \begin{bmatrix}
	2 \\ 1
\end{bmatrix} x + 
\begin{bmatrix}
	-1 \\ 1
\end{bmatrix} y =
\begin{bmatrix}
	1 \\ 5
\end{bmatrix}$$

\paragraph{Example} 3 by 3 system:

Each row represents a plane:

\begin{tabular}{c}
	2u + v + w = 5 \\
	4u - 6v + 0 = -2 \\
	-2u + 7v + 2w = 9
\end{tabular}

Remember: inner product of vector with another vector equals 0 $\Rightarrow$ orthogonal.

Column interpretation:

\[
	\begin{bmatrix}
		2 \\ 4 \\ 2
	\end{bmatrix} u + 
	\begin{bmatrix}
		1 \\ -6 \\ 7
	\end{bmatrix} v + 
	\begin{bmatrix}
		1 \\ 0 \\ 2
	\end{bmatrix} w =
	\begin{bmatrix}
		5 \\ -2 \\ 9
	\end{bmatrix}
\]

\paragraph{Example} Overdetermined system:
\[
	\begin{bmatrix}
		1 & 1 \\
		2 & 3 \\
		3 & 4 \\
	\end{bmatrix}
	\begin{bmatrix}
		c \\ d
	\end{bmatrix} = 
	\begin{bmatrix}
		2 \\ 5 \\ 7
	\end{bmatrix}
\]

Solution: $c = 1, d = 1$

In 4 dimensions, the rows represent 3-spaces, which are `flat' relative to 4 dimensional space. If we intersect $(x,y,z,t=0)$ with $(x,y,z=0,t)$, two three spaces, we get $(x,y)$ plane.

\[
	a_1 u + a_2 v + a_3 w + a_4 z = b
\]

\[
	A = (a_1 | a_2 | a_3 | a_4)
\]

\subsection{Algorithmic approach}
Generalizing to $n$ by $n$. How to solve $Ax = b$ in a way that scales well? Gaussian elimination (row reduction).


\begin{tabular}{c}
	$2u + v + w = 5$ \\
	$4u - 6v + 0 = -2$ \\
	$-2u + 7v + 2w = 9$ \\\hline
\end{tabular}

$\Rightarrow$
\begin{tabular}{c}
	$2u + v + w = 5$ \\
	   $-8v - 2w = -12$ \\
         $8v + 3w = 14$ \\\hline
\end{tabular}

$\Rightarrow$
\begin{tabular}{c}
	$2u + v + w = 5$ \\
	   $-8v - 2w = -12$ \\
         $w = 2$ \\\hline
\end{tabular}

$\Rightarrow v = 1, u = 1$

We need a process that takes:

\[
	A = \begin{bmatrix}
		2 & 1 & 1 \\
		4 & -6 & 0 \\
		-2 & 7 & 2 \\
	\end{bmatrix}
\]
\[
	x = \begin{bmatrix}
		u \\ v \\ w
	\end{bmatrix}
\]
\[
	b = \begin{bmatrix}
		5 \\ -2 \\ 9
	\end{bmatrix}
\]

\dots this $Ax=b$ problem and transforms it to a $Ux = \hat{b}$ problem. We can get an upper triangular matrix, and obtain solution by back substitution.

\paragraph{Problems} One issue that could arise is if the bottom row is all 0s: infinitely many solutions.

%!TEX root = ../notes.tex
\section{September 11 Lecture}

Last class:
\begin{itemize}
	\item Introduced first central problem of linear algebra: solving linear equations
	\item Studied column and row interpretation of linear systems
	\item Introduced Gaussian elimination
\end{itemize}

\paragraph{Example} (from previous class) Row/Column interpretation.

\begin{tabular}{c}
	2u + v + w = 5 \\
	4u - 6v + 0 = -2 \\
	-2u + 7v + 2w = 9
\end{tabular}

Row: Three planes intersecting. Column: linear combination of three vectors

\[
	A = \begin{bmatrix}
		2 & 1 & 1 \\
		4 & -6 & 0 \\
		-2 & 7 & 2 \\
	\end{bmatrix}
\]

We were trying to figure out how to transform matrix $A$ into an upper triangular matrix.

\[
	\begin{bmatrix}
		1 & 0 & 0 \\
		-2 & 1 & 0 \\
		0 & 0 & 1 \\
	\end{bmatrix}
	\begin{bmatrix}
		5 \\ -2 \\ 9
	\end{bmatrix}
	=
	\begin{bmatrix}
		5 \\ -12 \\ 9
	\end{bmatrix}
\]

\[
	\begin{bmatrix}
		1 & 0 & 0 \\
		-2 & 1 & 0 \\
		0 & 0 & 1 \\
	\end{bmatrix}
	\begin{bmatrix}
		2 & 1 & 1 \\
		4 & -6 & 0 \\
		-2 & 7 & 2 \\
	\end{bmatrix} = 
	\begin{bmatrix}
		2 & 1 & 1 \\
		0 & -8 & -2 \\
		-2 & 7 & 2 \\
	\end{bmatrix}
\]

\paragraph{Matrix operations} Addition is associative: $A+B+C = (A+B) + C = A+ (B+C)$

Multiplication: dimension $m \times n$ multiplied by $n \times p$ results in $m \times p$ matrix. $AB \neq BA$.

\[
	\begin{bmatrix}
		0 & 1 \\ 1 & 0\\
	\end{bmatrix}
	\begin{bmatrix}
		2 & 3 \\ 7 & 8 \\ 
	\end{bmatrix} =
	\begin{bmatrix}
		7 & 8 \\ 2 & 3
	\end{bmatrix}
\]

\[
	\begin{bmatrix}
		2 & 3 \\ 7 & 8 \\ 
	\end{bmatrix} 
	\begin{bmatrix}
		0 & 1 \\ 1 & 0\\
	\end{bmatrix}=
	\begin{bmatrix}
		3 & 2 \\ 8 & 7
	\end{bmatrix}
\]

Matrix multiplication:
\[
	\begin{bmatrix}
	a_{11} & a_{12} & a_{13} & ... & a_{1n} \\
	a_{21} & a_{22} & a_{23} & ... & a_{2n} \\
	... & &  & ... &  \\
	a_{m1} & a_{m2} & a_{m3}& ... & a_{mn} \\
	\end{bmatrix}
	\begin{bmatrix}
		x_1 \\ x_2 \\ ... \\ x_n
	\end{bmatrix} = 
	\begin{bmatrix}
		\sum_{i=1}^n a_{1i}x_i \\
		\sum_{i=1}^n a_{2i}x_i \\ 
		... \\
		\sum_{i=1}^n a_{ni}x_i\\
	\end{bmatrix}
\]

\[
	\begin{bmatrix}
		| & | &  & | \\
		a_1 & a_2 & ... & a_n \\
		| & | &  & | \\
	\end{bmatrix}
		\begin{bmatrix}
		x_1 \\ ... \\ x_n
	\end{bmatrix} = 
	a_1x_1+a_2x_2+...+a_nx_n
\]

\[
	\begin{bmatrix}
		| & | &  & | \\
		a_1 & a_2 & ... & a_n \\
		| & | &  & | \\
	\end{bmatrix}
	\begin{bmatrix}
		| &| \\
		b_1 & b_2 \\
		| & | \\
	\end{bmatrix} = 
	\begin{bmatrix}
		| & | \\
		Ab_1 & Ab_2 \\
		| & | \\
	\end{bmatrix} 
\]

\paragraph{Row reduction} In matrix form

\[
	A = \begin{bmatrix}
		2 & 1 & 1 \\
		4 & -6 & 0 \\
		-2 & 7 & 2 \\
	\end{bmatrix}
\]

\begin{enumerate}
	\item Subtract 2 times row 1 to row 2
	\[
		\begin{bmatrix}
			1 & 0 & 0 \\
			-2 & 1 & 0 \\
			0 & 0 & 1 \\
		\end{bmatrix}_{E_{21}}
		\begin{bmatrix}
			2 & 1 & 1 \\
			4 & -6 & 0 \\
			-2 & 7 & 2 \\
		\end{bmatrix}_A = 
		\begin{bmatrix}
			2 & 1 & 1 \\
			0 & -8 & -2 \\
			-2 & 7 & 2 \\
		\end{bmatrix}
	\]
	\item Subtract -1 times row 1 to row 3
	\[
		\begin{bmatrix}
			1 & 0 & 0 \\
			0 & 1 & 0 \\
			1 & 0 & 1 \\
		\end{bmatrix}_{E_{31}}
		E_{21}A = 
		\begin{bmatrix}
			2 & 1 & 1 \\
			0 & -8 & -2 \\
			0 & 8 & 3 \\
		\end{bmatrix}
	\]
	\item Subtract -1 times row 2 to row 3
	\[
		\begin{bmatrix}
			1 & 0 & 0 \\
			0 & 1 & 0 \\
			0 & 1 & 1 \\
		\end{bmatrix}_{E_{32}}
		E_{31}E_{21}A = 
		\begin{bmatrix}
			2 & 1 & 1 \\
			0 & -8 & -2 \\
			0 & 0 & 1 \\
		\end{bmatrix}
	\]
\end{enumerate}

Originally we wanted to solve $Ax=b$. Now we have:

\[
	E_{32}E_{31}E_{21}A = U
\]

where $U$ is an upper triangular matrix.

\[
	E_{32}E_{31}E_{21}Ax = Ux
\]

Let's let $E_{32}E_{31}E_{21} = L$. Then, we have

\[
	\begin{matrix}
		L^{-1}A=U \\
		A = LU \\
		Ux = C = E_{32}E_{31}E_{21}b
	\end{matrix}
\]

Now we can solve by back substitution.

\[
	L^{-1}=E_{32}E_{31}E_{21} = \begin{bmatrix}
		1 & 0 & 0 \\
		-2 & 1 & 0 \\
		-1 & 1 & 1 
	\end{bmatrix}
\]


Matrix inverse properties:

\[
	\begin{matrix}
		(AB)^{-1} = B^{-1}A^{-1} \\
		(A_1A_2...A_n)^{-1} = A_n^{-1}...A_2^{-1}A_1^{-1}
	\end{matrix}
\]

So we have:

\[
	L = \begin{bmatrix}
		1 & 0 & 0 \\
		2 & 1 & 0 \\
		-1 & -1 & 1
	\end{bmatrix} = E_{21}^{-1}E_{31}^{-1}E_{32}^{-1}
\]

\paragraph{Row reduction matrices} A matrix that subtracts $l$ times row $j$ from row $i$ is such that it includes $-l$ in row $i$, column $j$.

\[
	\begin{matrix}
			A = \begin{bmatrix}
		1 & 0 & 0 \\
		2 & 1 & 0 \\
		-1 & -1 & 1
	\end{bmatrix}_L
	\begin{bmatrix}
		2 & 1 & 1 \\
		0 & -8 & -2 \\
		0 & 0 & 1 \\
	\end{bmatrix}_U
	\end{matrix}
\]

$L$ is lower triangular and $U$ is upper triangular.

\begin{enumerate}
	\item Compute LU factorization
	\item Solve for $c$ in $Lc = b$ (forward substitution)
	\item Solve for $x$ in $Ux = c$ (back substitution)
\end{enumerate}

We want to solve $Ax=b$. We factor to get $LUx =b$. First we find $c$ such that $Lc = b$

\subsection{General Example}
\[
	\begin{bmatrix}
		l_{11} & 0 & 0 \\
		l_{21} & l_{22} & 0 \\
		l_{31} & l_{32} & l_{33}
	\end{bmatrix}
	\begin{bmatrix}
		c_1 \\ c_2 \\ c_3
	\end{bmatrix} = 
	\begin{bmatrix}
		b_1 \\ b_2 \\ b_3
	\end{bmatrix} \Rightarrow
	\begin{matrix}
		c_1 = b_1 / l_{11} \\
		c_2 = b_2 - b_1 l_{21}/ l_{11} \\
		c_3 = b_3 - l_{31}b_1 - l_{32}(b_2 - b_1 l_{21})
	\end{matrix}
\]

\[
	\begin{bmatrix}
		u_{11} & u_{12} & u_{13} \\
		0 & u_{22} & u_{23} \\
		0 & 0 & u_{33}
	\end{bmatrix}
	\begin{bmatrix}
		x_1 \\ x_2 \\ x_3
	\end{bmatrix} = 
	\begin{bmatrix}
		c_1 \\ c_2 \\ c_3
	\end{bmatrix} \Rightarrow
	\begin{matrix}
		x_3 = c_3/u_{33} \\
		x_2 = \frac{1}{u_{22}}(c_2 - u_{23}c_3/u_{33})\\
		x_1 = .....
	\end{matrix}
\]
%!TEX root = ../notes.tex
\section{September 13 Lecture}

Announcements

\begin{itemize}
  \item Matlab tutorials (sections)
  \item Final projects
  \begin{itemize}
    \item Adjustment based on class size
    \item Pairs
  \end{itemize}
  \item Assignment 1 due Fri @ 7pm in Pierce 303
  \item Collaboration policy
\end{itemize}

From last time:

\begin{itemize}
  \item Linear equations $\rightarrow$ Matrix notation
  \item Column $j$ of $AB=Ab_j$
  \[
    A
    \begin{bmatrix}
      | & | &  & | \\
      b_1 & b_2 & ... & b_n \\
      | & | &  & | \\
    \end{bmatrix} = 
    \begin{bmatrix}
      Ab_1 & Ab_2 & ... & Ab_n \\
    \end{bmatrix} 
  \]
  \item Introduced the $LU$ factorization of square matrix $A$ (see general example at end of last lecture)
  \[
    Ax=b \Rightarrow LUx = b
  \]
\end{itemize}

\begin{enumerate}
  \item Find $LU$
  \item Solve for $c$ in $Lc=b$
  \item Solve for $x$ in $Ux=c$
\end{enumerate}

\paragraph{Example} $LU$ factorization

\[
  A = 
  \begin{bmatrix}
    1 & 0 & 1 \\ 
    2 & 2 & 2 \\
    3 & 4 & 5
  \end{bmatrix}
\]

\begin{enumerate}
  \item Subtract 2 times row 1 to row 2
  \[
    \begin{bmatrix}
      1 & 0 & 0 \\ 
      -2 & 1 & 0 \\
      0 & 0 & 1
    \end{bmatrix}_{E_{21}} A =
    \begin{bmatrix}
      1 & 0 & 1 \\ 
      0 & 2 & 0 \\
      3 & 4 & 5
    \end{bmatrix}
  \]
  \[
    E_{21}^{-1} = \begin{bmatrix}
      1 & 0 & 0 \\ 
      2 & 1 & 0 \\
      0 & 0 & 1
    \end{bmatrix}
  \]
  \item Subtract 3 times row 1 to row 2
  \[
    \begin{bmatrix}
      1 & 0 & 0 \\ 
      0 & 1 & 0 \\
      -3 & 0 & 1
    \end{bmatrix}_{E_{31}} E_{21} A =
    \begin{bmatrix}
      1 & 0 & 1 \\ 
      0 & 2 & 0 \\
      0 & 4 & 2
    \end{bmatrix}
  \]
  \item Subtract 2 times row 2 to row 3
  \[
    \begin{bmatrix}
      1 & 0 & 0 \\ 
      0 & 1 & 0 \\
      0 & -2 & 1
    \end{bmatrix}_E{32} E_{31} E_{21} A =
    \begin{bmatrix}
      1 & 0 & 1 \\ 
      0 & 2 & 0 \\
      0 & 0 & 2
    \end{bmatrix}_U
  \]
\end{enumerate}

\[
  L^{-1} = E_{32} E_{31} E_{21} = \begin{bmatrix}
    1 & 0 & 0
    -2 & 1 & 0
    -3 & -2 & 1
  \end{bmatrix}
\]

\[
  \begin{matrix}
    L^{-1}A = U \\
    L = E_{21}^{-1} E_{31}^{-1} E_{32}^{-1} \\
    L = \begin{bmatrix}
      1 & 0 & 0 \\
      2 & 1 & 0 \\
      3 & 2 & 0 \\
    \end{bmatrix}
  \end{matrix}
\]

\paragraph{Generalizing $LU$ factorization} To $n \times n$ matrix:
\[
  \begin{bmatrix}
    a_{11} & a_{12} & a_{13} & ... &  a_{1n} \\
    a_{21} & a_{22} & a_{23} & ... &  a_{2n} \\
    ... & ... & ... & ... & ...  \\
    a_{n1} & a_{n2} & a_{n3} & ... &  a_{nn} \\
  \end{bmatrix}
\]

\begin{enumerate}
  \item Introduce zeros below $a_{11}$ by subtracting multiples of row 1
  \item Use multipliers $l = \frac{a_{i1}}{a_{11}}$
  \item Repeat 1 and 2 for $a_{22}^*,a_{33}^*$, ...
\end{enumerate}

Step 1:
\[
  \begin{array}{c|cccc}
    a_{11} & a_{12} & a_{13} & ... &  a_{1n} \\\hline
    0 & a_{22}^* & a_{23}^* & ... &  a_{2n}^* \\
    ... & ... & ... & ... & ...  \\
    0 & a_{n2}^* & a_{n3}^* & ... &  a_{nn}^* \\
  \end{array}
\]

Step 2:
\[
  \begin{array}{c|c|ccc}
    a_{11} & a_{12} & a_{13} & ... &  a_{1n} \\\hline
    0 & a_{22}^* & a_{23}^* & ... &  a_{2n}^* \\\hline
    ... & 0 & a_{33}^* & ... & a_{3n}^* \\
    ... & ... & ... & ... & ...  \\
    0 & 0 & a_{n3}^* & ... &  a_{nn}^* \\
  \end{array}
\]

How many operations does this algorithm use?

\[
  \sum_{k=1}^n k^2 - \sum_{k=1}^n k = \frac{n(n+1)(2n+1)}{6} - \frac{n(n+1)}{2}
\]
%!TEX root = ../notes.tex
\section{September 18 Lecture}

To review: solving $Ax=b$:
\begin{enumerate}
	\item Find $LU=A$
	\item Solve for $c$ in $Lc=b$ (forward substitution)
	\item Solve for $x$ in $Ux=c$ (back-subst)
\end{enumerate}

Multipliers to find $U$ are entries of $L$.

What is the \# of operations needed to get $LU$ factorization?

\[
	\approx {n^3 - n \over 3}
\]

\paragraph{Forward substitution} Number of operations:

$(n-1) + (n-2) + ... (1) \approx O(n^2)$

Back substitution is similar process (also $O(n^2)$). Most time consuming place is step 1.

\paragraph{Algorithm Failure} This $Ax=b$:

\[
	\begin{bmatrix}
		0 & 1 \\ 1 & 0
	\end{bmatrix}
	\begin{bmatrix}
		x_1 \\ x_2
	\end{bmatrix} = 
	\begin{bmatrix}
		1 \\ 1
	\end{bmatrix}
\]

has solution $\begin{bmatrix}
	1 \\ 1
\end{bmatrix}$. However, our algorithm won't find the answer because it can't switch rows. If the algorithm fails we have two options:

\begin{enumerate}
	\item We need to rearrange rows
	\item No solution
	\item Infinitely many solutions
\end{enumerate}

Example of (2):

\[
	\begin{bmatrix}
		0 & 1 \\ 0 & 0
	\end{bmatrix}
	\begin{bmatrix}
		x_1 \\ x_2
	\end{bmatrix} = 
	\begin{bmatrix}
		1 \\ 1
	\end{bmatrix}
\]

Example of (3):

\[
	\begin{bmatrix}
		0 & 1 \\ 0 & 1
	\end{bmatrix}
	\begin{bmatrix}
		x_1 \\ x_2
	\end{bmatrix} = 
	\begin{bmatrix}
		1 \\ 1
	\end{bmatrix}
\]

\paragraph{Fact} $\det(A) = \det(LU) = \det(L)\det(U)$

\[
	\det(U) = \prod_{i=1}^n u_{ii}
\]

\paragraph{Example} Consider this:

\[
	\begin{bmatrix}
		0.0001 & 1 \\ 1 & 1
	\end{bmatrix}
	\begin{bmatrix}
		x_1 \\ x_2
	\end{bmatrix} = 
	\begin{bmatrix}
		1 \\ 2
	\end{bmatrix}
\]

\[
	\begin{bmatrix}
		0.0001 & 1 \\ 0 & -9999
	\end{bmatrix}
	\begin{bmatrix}
		x_1 \\ x_2
	\end{bmatrix} = 
	\begin{bmatrix}
		1 \\ -9998
	\end{bmatrix}
\]

\[
	\Rightarrow x_2 = {9998 \over 9999}
\]

\[
	0.0001 x_1 + {9998 \over 9999} = 1
\]

\[
	\Rightarrow x_1= {10000 \over 9999}
\]

If we do all of this with limited precision (say 3 digits), we do the following:

\[
	\begin{bmatrix}
		0.0001 & 1 \\ 0 & -10^4
	\end{bmatrix}
	\begin{bmatrix}
		x_1 \\ x_2
	\end{bmatrix} = 
	\begin{bmatrix}
		1 \\ -10^4
	\end{bmatrix}
\]

\[
	\Rightarrow x_2 = 1
\]

Then if we use the first equation, we get

\[
	\Rightarrow x_1 = 0
\]

This is called \textbf{catastrophic cancellation}.
%!TEX root = ../notes.tex
\section{September 20 Lecture}
First part of AM120 is to solve $Ax=b$ for arbitrary $\norm{A} = n$.

\[
	u_{11} = a_{11}, u_{22} = a_{22} 
\]

Pseudocode did not have 0s in $L$ and $U$. Second part of code is given $L$ and $b$, should output $c$. Third part takes $U$ and $c$ and outputs $x$.

Assignment 2 Due on Monday morning (9am).

This Doolittle algorithm can fail:
\begin{enumerate}
	\item If there is a pivot = 0
	\begin{enumerate}
		\item System is singular $\Rightarrow \det(A) = 0$. This means there is no solution or infinitely many solutions.
		\item We can exchange rows and `cure' system.
		\[
			\det(A) = \det(L) \det(U) = 1 \prod_{k=1}^n u_{kk}
		\]
	\end{enumerate}
\end{enumerate}


\paragraph{Example} From last class:

\[
	\begin{bmatrix}
		0.0001 & 1 \\ 1 & 1
	\end{bmatrix}
	\begin{bmatrix}
		x_1 \\ x_2
	\end{bmatrix} = 
	\begin{bmatrix}
		1 \\ 2
	\end{bmatrix}
\]

This had true solution:

\[
		x_1 = {10000 \over 9999}, x_2 = {9998 \over 9999}
\]

But with limited precision (three digit arithmetic), we got'

\[
		x_1 = 0, x_2 = 1
\]

What if we switch the rows?

\[
	\begin{bmatrix}
		1 & 1 \\ 0.0001 & 1
	\end{bmatrix}
	\begin{bmatrix}
		x_1 \\ x_2
	\end{bmatrix} = 
	\begin{bmatrix}
		2 \\ 1
	\end{bmatrix}
\]

\[
	\begin{bmatrix}
		1 & 0 \\ 10^{-4} & 1
	\end{bmatrix}_L
	\begin{bmatrix}
		1 & 1 \\ 0 & 1
	\end{bmatrix}
	\begin{bmatrix}
		x_1 \\ x_2
	\end{bmatrix} =
	\begin{bmatrix}
		2 \\ 1 - 2 \cdot 10^{-4}t
	\end{bmatrix}
\]


\subsection{Finite precision}

The computer represents a floating point number with a sign, exponent, and digits for the value itself. When we are talking about $n$-digit arithmetic, we are referring to the number of digits storing the value.

\[
	\begin{matrix}
		U = \text{max exponent} \\
		L = \text{lowest exponent} \\
		P = \text{mantissa number of digits} \\
		\beta = \text{base} \\
	\end{matrix}
\]

For example, for $L=-1, U=1, p=2$ and $\beta = 10$'

\[
	\text{(sign)} d_0 . d_1 d_2 ... d_p \times 10^{\text{exponent}}
\]

\[
	\begin{matrix}
		\text{largest} & 9.9 \times 10^1 = 99\\
		\text{smallest (non-zero)} & 1.0 \times 10^{-1} = 0.1
	\end{matrix}
\]

Examples of real values in floating point systems:

\[
	\begin{matrix}
		& \beta & P & L & U \\
		\text{IEEE} & 2 & 24 & -126 & 123 & \text{single} \\
					& 2 & 53 & -1022 & 1023 & \text{double}\\ 
		\text{HP}   & 10 & 12 & -499 & 499
	\end{matrix}
\]

What is the total number of floating point numbers?

\[
	2(\beta - 1) \beta^{p-1} (u-l+1) + 1
\]

Largest representable number:

\[
	(\beta - 1).(\beta - 1)...(\beta - 1) \cdot \beta^U
\]

Smallest number (absolute value):

\[
	\beta^L (\text{underflow})
\]

\paragraph{Machine precision} Note that the difference between the real number and the floating point number chosen depends on exponent.

\[
	\forall x \in R, \exists \text{fl}(x) = \hat{x} \text{ such that } \abs{x-\hat{x}} \leq \sum \abs{x}
\]

Note that this is not really true for all $x \in R$ -- only within a certain range. $\epsilon_{\text{mach}}$ is the largest number s.t. fl($1 + \epsilon_{\text{mach}}) > 1$.

Floating point numbers are not associative:

\[
	A+(B+C)) \not= (A+B)+C
\]
\section{September 25 Lecture}
Need to solve $Ax=b$. In theory, we can find $A^{-1}$. In practice, this is not necessary -- it takes too many operations!

\[
  A \begin{bmatrix}
    | & | & \cdots & |  \\
    x_1 & x_2 & \cdots & x_n  \\
    | & | & \cdots & |  \\
  \end{bmatrix} = 
  \begin{bmatrix}
    | & | & \cdots & |  \\
    e_1 & e_2 & \cdots & e_n  \\
    | & | & \cdots & |  \\
  \end{bmatrix} = 
\]

\begin{align*}
  A A^{-1} &= I \\
  x &= A^{-1}b
\end{align*}

In practice, we find $LU$. Then we solve $Lc = b$L

\begin{align*}
  Lc &= b \\
  c = L^{-1}b
\end{align*}

Then, we solve $Ux = c$:

\begin{align*}
  Ux &= c
  L^{-1} A x &= L^{-1} b
\end{align*}

This algorithm fails if any of the pivots are 0.

\paragraph{Singular matrix} A matrix is singular if $\det(A) \not= 0 \iff Ax = b$ has a unique solution.

\subsection{Floating point nonsense}

\[
  fl(1 + \epsilon) > 1 \Rightarrow \epsilon = {1 \over 2} \beta^{1 - \rho}
\]

Machine precision ($\epsilon_{\text{mach}}$): single ~$10^{-8}$; double ~$10^{-16}$.

\begin{align*}
  a &= 1.23456 \times 10^2
  a &= -1.23455 \times 10^{-2}
  a &= -1.11123 \times 10^{-3}
  a+b+c &= (1 \times 10^{-3}) -1.11123 \times 10^{-3} = -1.1123 \times 10^{-4}
\end{align*}

This is the result if we calculate $(a+b) + c$. Notice that if we use 6-digit arithmetic and calculate $a+(b+c)$, we get 0! Floating point arithmetic is not associative.

\[
  \sum_{n=1}^\infty {1 \over n} \to
\]

\begin{enumerate}
  \item ${1 \over n}$ < UFL
  \item $\sum_{n=1}^{k_2} {1 \over n}$ > OFL
  \item fl$\left(\left(\sum_{n=1}^{k_3} {1 \over n}\right) + {1 \over k_3 + 1} \right) = \sum_{n=1}^{k_3} {1 \over n}$
\end{enumerate}

\subsection{Catastrophic cancellation}

If we add a very small number to a big number, sometimes we get the same big number! Like $1.0 \times 10^8 + 1.0 \times 10^{-9} = 1.0 \times 10^8$.

\begin{align*}
  A + b &= A \\
  A - b &= A
\end{align*}

Partial pivoting minimized catastrophic cancellation. *Bunch of matrices about the steps of partial pivoting, that I don't even think were accurate*

\paragraph{Theorem} For a non-singular and square matrix $A$, $\exists P$ (permutation matrix) thta reorders rows of $A$ to avoid zeros in the pivot positions. $Ax=b$ has a unique solution and with the rows ordered ``in advance.''

\[
  PA = LU \text{ and } L \text{ and } U \text{ are unique}
\]
%!TEX root = ../notes.tex
\section{September 27 Lecture}

\paragraph{Last Class} Theory: For a non-singular and square matrix $A$, $\exists P$ (permutation matrix) that reorders rows of $A$ to avoid zeroes in the pivot positions

$Ax=b$ has a unique solution, and with the rows ordered ``in advance:''

\[
	P A = L U \text{ where } L \text{ and } U \text{ are unique.}
\]

Note: If $A$ is singular, no $P$can produce a full set of pivots and elimination fails.

Gaussian elimination with partial pivoting: if a pivot is zero, then $A$ is singular

\[
	A x = 
	\begin{bmatrix}
		0 & 4 & 1 \\
		1 & 1 & 3 \\
		2 & -2 & 1
	\end{bmatrix}
	\begin{bmatrix}
		x_1 \\ x_2 \\ x_3
	\end{bmatrix} = 
	\begin{bmatrix}
		9 \\ 6 \\ -1
	\end{bmatrix} = 
	b
\]

\[
	PA =
	\begin{bmatrix}
		2 & -2 & 1 \\
		0 & 4 & 1 \\
		1 & 1 & 3
	\end{bmatrix},
	P = 
	\begin{bmatrix}
		0 & 0 & 1 \\
		1 & 0 & 0 \\
		0 & 1 & 0
	\end{bmatrix}
\]

\[
	\tilde{A} = PA = LU = 
	\begin{bmatrix}
		1 & 0 & 0
		0 & 1 & 0
		1/2 & 1/2 & 1
	\end{bmatrix}_{\tilde{L}}
	\begin{bmatrix}
		2 & -2 & 1
		0 & 4 & 1
		0 & 0 & 6
	\end{bmatrix}_{\tilde{U}}
\]

% for k = 1:n-1
% 	find p such that |a(p,k)| > |a(i,k)| for k <= i <= n
%	if p != k then
%		interchange rows k and p (and record permutation)
%	if a_kk = 0 then STOP

\subsection{Ill-conditioned Matrices}

The presence of round-off error makes it difficult to identify singular matrices.

\[
	A =
	\begin{bmatrix}
		1000 & 999 \\ 999 & 998		
	\end{bmatrix}	
	\rightarrow
	\begin{bmatrix}
		1 & 0 \\ .999 & 1
	\end{bmatrix}_L
	\begin{bmatrix}
		1000 & 999 \\
		0 & -.001
	\end{bmatrix}_U
\]


\[
	Ax = b = \begin{bmatrix} 1999 \\ 1997 \end{bmatrix}
	\rightarrow
	x = \begin{bmatrix} 1 \\ 1 \end{bmatrix}
\]

With limited precision (5-digit arithmetic), $A$ appears singular, because 0.999(999)=998.00.

\[
	Ax = \hat{b} = \begin{bmatrix} 1998.99 \\ 1997.01 \end{bmatrix} =
	b + \delta b = b + 10^{-2} \begin{bmatrix} -1 \\ 1 \end{bmatrix}
	\Rightarrow
	x = \begin{bmatrix} 20.97 \\ -18.99 \end{bmatrix}
\]

Small change in $b$ and same $A$, $x$ changes a lot. We call $A$ `ill-conditioned,' meaning that its `condition number,' $k(A)$ is big. This is the definition of condition number:

\[
	{\norm{\delta x} \over \norm{x}} \leq k(A) {\norm{\delta b} \over \norm{b}}
\]

Hilbert matrices are of this kind: changing $b$ a little bit, $x$ changes a lot (they are ill conditioned). Condition number of a singular matrix $A$ is $\infty$.

If ${\norm{\delta x} \over \norm{x}} > 1$ we don't expect to find a solution close to the one we were looking for.

In numberical calculations, singular matrices are indistinguishable from ill-conditioned matrix.


\[
	\begin{matrix}
		-{d^2 u \over dx^2} = f(x), 0 \leq x \leq 1 \\
		u(0) = c_1
		u(1) = c_2
	\end{matrix}
\]

\[
	u(x+h) = u(x) + h u'(x) + h^2 {u''(x) \over 2} + \dots + h^k {d^k u \over d x^k}
\]

Discretize interval into points $x_i$, solve for value of $u$ at each point. At each point we can solve the problem as a linear algebra problem. We ignore higher terms, and say:

\[
	u'(x_i) \approx {u(x_i+h) - u(x_i) \over h}
\]

\[
	-u''(x_i) \approx -{u(x_{i+1} - 2 u(x_i) + u(x_{i-1})) \over h^2 } = f(x_i)
\]

This results in a large matrix:

\[
	\begin{bmatrix}
		2 & -1 \\
		-1 & 2 & -1 \\
		0 & -1 & 2 & -1 \\
		 & & & & \ddots \\ \\ \\
	\end{bmatrix}
	\begin{bmatrix}
		u_1 \\ u_2 \\ \vdots \\ u_m
	\end{bmatrix}
	=
	\begin{bmatrix}
		f(x_1) \\ \\ \vdots \\ f(x_m)
	\end{bmatrix}
\]

%!TEX root = ../notes.tex
\section{October 2 Lecture}

\paragraph{Last class} Condition numbers. Blah blah blah blah blah.

We have implemented solving $Ax = b$ the same way that MATLAB's ``$\backslash$'' function works. Now we move on to other things.

\subsection{Over/underconstrained Systems}

\[
	-{d^2 u \over dx^2} = f(x), u(0) = \alpha, u(1) = \beta
\]

Approximation of second derivative at discrete point $x_i$:

\[
	-{d^2 u (x_i) \over dx^2} \approx {u(x_{i+1}) - 2u(x_i)+u(x_{i-1})) \over h^2}
\]

\[
	h^2 f(x_i) \approx u(x_{i+1}) - 2u(x_i)+u(x_{i-1}))
\]

At the boundary:

\[
	h^2 f(x_1) \approx u(x_{2}) - 2u(x_1)+\alpha
\]

We got this from this, ignoring smaller terms:

\[
	u(x+h) = u(x) + {du \over dx}(x) h + {d^2u \over dx^2} {h^2 \over 2} + ... 
\]

This yields:

\[
	\begin{bmatrix}
		2 & -1 \\
		-1 & 2 & -1 \\
		0 & -1 & 2 & -1 \\
		 & & & & \ddots \\ \\ \\
	\end{bmatrix}
	\begin{bmatrix}
		u(x_1) \\ u(x_2) \\ \vdots \\ u(x_m)
	\end{bmatrix}
	=
	\begin{bmatrix}
		-\alpha + f(x_1)h^2 \\ f(x_2)h^2\\ \vdots \\ -\beta + f(x_m)h^2
	\end{bmatrix}
\]

Need to solve $Ax = b$. Now we are studying methods that have to do with matrices that are not square! First, we'll say $m < n$:

\[
	\begin{bmatrix}
		& & & & & & \\
		& & & A & & & \\
		& & & & & &
	\end{bmatrix}_{m \times n}
	\begin{bmatrix}
		\\ \\ \\ x \\ \\ \\ \\
	\end{bmatrix}_{n \times 1} =
	\begin{bmatrix}
		\\ b \\ \\
	\end{bmatrix}_{m \times 1}
\]

The left side of $A$ has an $m \times m$ square section. This is underdetermined, so there are many solutions. What is $m > n$:

\[
	\begin{bmatrix}
		\\
		 \\
		& A &\\
		\\
		\\
	\end{bmatrix}_{m \times n}
	\begin{bmatrix}
		\\ x \\ \\
	\end{bmatrix}_{n \times 1} =
	\begin{bmatrix}
		\\ \\ \\ b \\ \\ \\ \\
	\end{bmatrix}_{m \times 1}
\]

This is an overconstrained system, which may not have a solution. A real example of this is fitting a line to points in a least-squares sense. If each point is at $(t_i, y_i)$, and we want to find a line $y = mt + b$, then we solve:

\[
	\begin{bmatrix}
		t_1 & 1\\
		t_2 & 1\\
		\vdots \\
		t_m & 1
	\end{bmatrix}_A
	\begin{bmatrix}
		m \\ b
	\end{bmatrix}_x =
	\begin{bmatrix}
		y_1 \\ y_2 \\ \vdots \\ y_m
	\end{bmatrix}_b
\]

\subsection{Vector Spaces}
\textbf{Vector spaces} have two operations:

[Example: $\bbR^n$]

\begin{enumerate}
	\item if $x,y \in \bbR^n$, $x+ y = z \in \bbR^n$
	\item if $x \in \bbR^n, \alpha \in \bbR$, $\alpha x \in \bbR^n$
\end{enumerate}

These properties must hold:

\textbf{Addition}: for $x, y, z \in \bbR^n$
\begin{enumerate}
	\item Commutativity: $x + y = y + x$
	\item Associativity: $x + (y + z) = (x + y) + z$
	\item $\exists!$\footnote{`!' indicates there exists some unique zero vector} zero vector s.t. $x + 0 = x \forall x \in \bbR^n$
	\item $\exists! -x \in \bbR^n$ s.t. $x + (-x) = 0 \forall x \in \bbR^n$

	\textbf{Scalar multiplication}

	\item $1 x = x$ : 1 is scalar
	\item $(c_1 c_2) x = c_1(c_2x)$
	\item $c(x+y) = cx + cy$
	\item $(c_1 + c_2)x = c_1 x + c_2 x$
\end{enumerate}

A \textbf{subspace} is a non-empty subset of a vector space that satisfies all these properties and all linear combinations stay in the subspace.

\paragraph{Example}

\[
	\begin{bmatrix}
		1 & 0 \\ 5 & 4 \\ 4 & 4 \\
	\end{bmatrix}
	\begin{bmatrix}
		u \\ v
	\end{bmatrix} 
	=
	\begin{bmatrix}
		1 \\ 5 \\ 4
	\end{bmatrix} u
	\begin{bmatrix}
		0 \\ 4 \\ 4
	\end{bmatrix} v
\]
\section{October 4 Lecture}

Use `$\backslash$' operator in MATLAB from now on unless we specify otherwise. Assignment 4 is now due Tuesday in class.

\paragraph{Last class} We discussed vector spaces, with the intention of being able to solve any linear algebra problem, whether it be overdetermined or underdetermined. Vector spaces have two operations, addition and scalar multiplication, with 8 axioms. Note: no notion of proximity or distance (no topology).

\paragraph{Subspace} A non-empty subset of a vector space. Closed under addition and scalar multiplication. ($x+y \in$ subspace, $ax \in$ subspace).

\[
	\begin{matrix}
		\bbR^2 & \text{smallest sub-space} & \text{\{zero\} element} \\
		 & \begin{bmatrix}
		 	0 \\ 0
		 \end{bmatrix} \\
		 & \text{largest sub-space} & \bbR^2 \\\hline
		 \bbR^3 & \begin{bmatrix}
		 	0 \\ 0 \\0 
		 \end{bmatrix} \\
		 & \text{planes and lines that go through } \begin{bmatrix}
		 	0 \\ 0 \\ 0
		 \end{bmatrix} \text{ are subspaces.}
	\end{matrix}
\]

\[
	A = \begin{bmatrix}
		1 & 0 \\ 5 & 4 \\ 2 & 4
	\end{bmatrix},
	x = \begin{bmatrix}
		x_1 \\ x_2
	\end{bmatrix}, 
	Ax = \begin{bmatrix}
		1 \\ 5 \\ 2 
	\end{bmatrix} x_1 + 
	\begin{bmatrix}
		0 \\ 4 \\ 4
	\end{bmatrix} x_2
\]

Column spaces of $A$ (denoted $C(A)$) is the spaces that contains all linear combinations of the columns of $A$.

If we have a matrix $\begin{bmatrix}A\end{bmatrix}_{M \times N}$ (with $m$ rows, $n$ columns), then $C(A) \in \bbR^m$.

$b$ and $\tilde{b} \in C(A), \exists x$ and $\tilde{x}$

\[
	\begin{matrix}
		Ax = b & A(x+\tilde{x}) = b + \tilde{b}\\
		A \tilde{x}  = \tilde{b} \\
		c b & A(cx) = cAx = cb
	\end{matrix}
\]	

Null space:

\[
	A = \begin{bmatrix}
		1 & 0 \\ 0 & 1 \\ 0 & 0
	\end{bmatrix}
\]	

Null space of $A$ consists of all vectors $x$ such that $Ax = 0$, denoted $N(A) \in \bbR^n$.

\[
	Ax = \begin{bmatrix}
		1 \\ 5 \\ 2
	\end{bmatrix} x_1 + 
	\begin{bmatrix}
		0 \\ 4 \\ 4
	\end{bmatrix} x_2 = 
	\begin{bmatrix}
		0 \\ 0 \\ 0
	\end{bmatrix}
\]

\[
	\begin{bmatrix}
		x_1 \\ x_2
	\end{bmatrix} \in N(A),
	\begin{bmatrix}
		0 \\ 0
	\end{bmatrix} \in N(A)
\]

\paragraph{Theorem} If zero is the only element of $N(A) \Rightarrow$ columns of $A$ are linearly independent.

If $N(A) = \{0\}$ and $A$ is a square matrix $\Rightarrow \exists! x$ such that $Ax = b$ for any $b$.

Basis for a \textbf{vector space} $V \{v_k\}$.

\begin{enumerate}
	\item $v_k$'s are linearly independent
	\item they span $V$ (any $v \in V$ is a linear combination of the basis vectors $\{v_k\}$)
\end{enumerate}

\[
	\Rightarrow \exists! \text{ way to represent any element of } V
\]

\[
	text{dim}(V) = \text{ \# of basis vectors}
\]

The \textbf{complete solution} of a linear system of equations

\[
	 Ax=b \text{ is given by } x = x_p + x_n \text{ (if it exists)}
\]

where

\[
	Ax_p = b and Ax_n = 0
\]

\[
	A(x_p + x_n) = b + 0 = b
\]

\[
	\begin{matrix}
		A = \begin{bmatrix}
			1 & 2 \\ 2 & 4
		\end{bmatrix} &
		C(A) = k \begin{bmatrix}
			1 \\ 2
		\end{bmatrix} \text{ line}\\
		& N(A) = d \begin{bmatrix}
			2 \\ -1
		\end{bmatrix} \text{ line}
	\end{matrix}
\]

\[
	\begin{bmatrix}
		1 & 2 \\ 2 & 4
	\end{bmatrix}
	\begin{bmatrix}
		x_1 & x_2
	\end{bmatrix} = Ax  = b = \begin{bmatrix}
		3 \\ 6
	\end{bmatrix}
\]

\[
	x = \begin{bmatrix}
		3 \\ 0
	\end{bmatrix}_{x_{p}} + d \begin{bmatrix}
		2 \\ -1
	\end{bmatrix}_{x_{n}}
\]

\paragraph{Theorem}: For any $m \times n$ matrix $A \exists P$ (permutation) and $L$ lower triangular matrix and an $m \times n$ Echelon matrix $U$ such that $PA = LU$

\[
	A = \begin{bmatrix}
		1 & 3 & 3 & 2 \\
		2 & 6 & 9 & 7 \\
		-1 & -3 & 3 & 4
	\end{bmatrix}
\]

Let's find $LU$:

\[
	L = \begin{bmatrix}
		1 & 0 & 0\\
		2 & 1 & 0\\
		-1 & 2 & 1 \\
	\end{bmatrix},
	U = \begin{bmatrix}
		1 & 3 & 3 & 2 \\
		0 & 0 & 3 & 3 \\
		0 & 0 & 0 & 0
	\end{bmatrix}
\]

Note that $L$ is $m \times m$ square, and $U$ is also $m \times n$. Also, we see that $U$ has 2 LI columns.

We already knew that the column space of $A$ lives in $\bbR^3$, so one column had to be dependent. Now we know that only two vectors in the 4 columns are LI, so the column space is a plane. Null space is also a plane, but it lives in $\bbR^4$.

Another example:

\[
	\begin{matrix}
		(t_1, y_1)' = (0,0) \\
		(t_2, y_2) = (2,1)\\
		\begin{bmatrix}
			1 & 0 \\ 1 & 2
		\end{bmatrix}_A
		\begin{bmatrix}
			x_1 \\ x_2
		\end{bmatrix} = 
		\begin{bmatrix}
			0 \\ 1
		\end{bmatrix}
	\end{matrix}
\]

\[
	\begin{matrix}
		y = P(t) = x_1 + x_2 t \\
		y_1 = P(t_1) = x_1 + x_2 t_1 \\
		y_2 = P(t_2) = x_1 + x_2 t_2
	\end{matrix}
\]

We get $P = {1 \over 2 } t$
\section{October 9 Lecture}

The transpose of a matrix $A$ (denoted $A^T$) is a matrix with columns directly from rows of $A$ (the $i$th row becomes the $i$th column of $A^T$). $(AB)^T = B^T A^T$.

\paragraph{Colummn space of $A$} Denoted $C(A)$. Contains all linear combinations of the columns of $A$. $C(A)$ is a subspace of $\bbR^m$.

\paragraph{Complete solution} to the problem $Ax=b$ can be expressed as $x=x_p + x_n$. $x_n \in N(A), x_p$ a particular solution.

\begin{align*}
  A &= \begin{bmatrix}
    1 & 2 \\ 2 & 4
  \end{bmatrix} \\
  A x_p &= \begin{bmatrix}
    3 \\ 6
  \end{bmatrix}
\end{align*}

Should know
\begin{enumerate}
  \item What is a vector space?
  \item When a set of vectors are linearly independent?
  \item Dimension of a vector space
  \item Basis for a vector space
\end{enumerate}

\paragraph{Theorem} $N(A)$ contains only the vector iff zero the columns of $A$ are linearly independant. Therefore, if $A$ is square, $\exists! x $for any$ b$.

\[
  A = \begin{bmatrix}
    1 & 3 & 3 & 2 \\
    2 & 6 & 9 & 7 \\
    -1 & -3 & 3 & 4
  \end{bmatrix} =
  \begin{bmatrix}
    1 & 0 & 0 \\
    2 & 1 & 0 \\
    -1 & 2 & 1 
  \end{bmatrix}_L
  \begin{bmatrix}
    1 & 3 & 3 & 2 \\
    0 & 0 & 3 & 3 \\
    0 & 0 & 0 & 0
  \end{bmatrix}_U
\]

The \textit{row space of $A$} is the column space of $A^T$ ($C(A^T)$). It is a subspace of $\bbR^n$.

The left null space of $A$ is the null space of $A^T$. $y \in N(A^T)$ if $A^Ty=0$ iff $y^T A = 0$.

Let's calculate the null space of $A$. We've broken $A$ into $LU$. Since $U$ was obtained by adding and subtracting rows of $A$, it follows that $N(A) = N(U)$.

\[
  \begin{bmatrix}
    1 & 3 & 3 & 2 \\
    0 & 0 & 3 & 3 \\
    0 & 0 & 0 & 0
  \end{bmatrix}_U 
  \begin{bmatrix}
    u \\ v \\ w \\y
  \end{bmatrix} = 
  \begin{bmatrix}
    0 \\ 0 \\0
  \end{bmatrix}
\]

\begin{align*}
  3w + 3y &= 0 \Rightarrow w = -y \\
  u + 3v - 3y + 2y &= 0 \\
  u &= y - 3v
\end{align*}

\[
  x \in N(A) = \begin{bmatrix}
    y - 3v \\ v \\ -y \\ y
  \end{bmatrix} = 
  y \begin{bmatrix}
    1 \\ 0 \\ -1 \\ 1
  \end{bmatrix} + 
  v \begin{bmatrix}
    -3 \\ 1 \\ 0 \\ 0
  \end{bmatrix}
\]

The linear combination of these two vectors generates a plane in $\bbR^4$, the null space of $A$.

\[
  Ax = \begin{bmatrix}
    b_1 \\ b_2 \\ b_3
  \end{bmatrix}
\]

\[
  \begin{bmatrix}
    1 & 3 & 3 & 2 \\
    0 & 0 & 3 & 3 \\
    0 & 0 & 0 & 0
  \end{bmatrix}_U 
  \begin{bmatrix}
    u \\ v \\ w \\y
  \end{bmatrix} = 
  \begin{bmatrix}
    b_1 \\ b_2 - 2b_1 \\ (b_3 + b_1) - 2(b_2 - 2b_1)
  \end{bmatrix} = 
    \begin{bmatrix}
    b_1 \\ b_2 - 2b_1 \\ b_3 - 2b_2 + 52b_1
  \end{bmatrix}
\]

\[
  \iff b_3 - 2b_2 + 5b_1 = 0 \text{ solvability condition}
\]

Now, we choose:

\[
  b = \begin{bmatrix}
    1 \\ 5 \\ 5
  \end{bmatrix},
  \tilde{b} = \begin{bmatrix}
    1 \\ 3 \\ 0
  \end{bmatrix}
\]

\[
  \begin{bmatrix}
    1 & 3 & 3 & 2 \\
    0 & 0 & 3 & 3 \\
    0 & 0 & 0 & 0
  \end{bmatrix}_U 
  \begin{bmatrix}
    u \\ v \\ w \\y
  \end{bmatrix} =
  \begin{bmatrix}
    1 \\ 3 \\ 0
  \end{bmatrix}
\]

\begin{align*}
  3w + 3y &= 3 \Rightarrow w = 1 - y \\
  u + 3v + 3(1-y) + 2y &= 1 \Rightarrow u = -2 - 3v + y
\end{align*}

Now we can write

\[
  \begin{bmatrix}
    1 & 3 & 3 & 2 \\
    0 & 0 & 3 & 3 \\
    0 & 0 & 0 & 0
  \end{bmatrix}_U 
  \begin{bmatrix}
    u \\ v \\ w \\y
  \end{bmatrix} =
  \begin{bmatrix}
    1 \\ 3 \\ 0
  \end{bmatrix} \Rightarrow
  x = \underbrace{\begin{bmatrix}
    -2 \\ 0 \\ 1 \\ 0
  \end{bmatrix}}_{x_p}+
  \underbrace{v \begin{bmatrix}
    -3 \\ 1 \\ 0 \\0
  \end{bmatrix} + 
  y \begin{bmatrix}
    1 \\ 0 \\ -1 \\ 1
  \end{bmatrix}}_{x_n}
\]

Remarks:
\begin{itemize}
  \item The null space of $A$, $N(A)$, and the row space of $A$, $C(A^T),$ are subspaces of $\bbR^n$.
  \item The left null space, $N(A^T)$, and column space of $A$, $C(A)$ are subspace of $\bbR^m$.
\end{itemize}

Transform $A \underbrace{\rightarrow}_\text{using Gauss elim} U$ we can immediately identify a basis for $C(A^T)$.

\[
  A = \begin{bmatrix}
    1 & 0 & 0 \\ 0 & 0 & 0
  \end{bmatrix},
  A^T = \begin{bmatrix}
    1 & 0 \\ 0 & 0 \\ 0 & 0
  \end{bmatrix}
\]
\[
  C(A) = \begin{bmatrix}
    1 \\ 0
  \end{bmatrix}d \text{ line in } \bbR^2
\]

Row space of $A$ = $\begin{bmatrix} 1 \\ 0 \\ 0 \end{bmatrix}$ c lives in $\bbR^3$

$N(A)$ lives in $\bbR^3: Ax = 0$:

\[
  A = \begin{bmatrix}
    1 & 0 & 0 \\ 0 & 0 & 0
  \end{bmatrix}
  \begin{bmatrix}
    x_1 \\ x_2 \\ x_3
  \end{bmatrix} = \vec{0}
\]

We see that $x_1$ must be 0, but the other values are free:

\begin{align*}
  x_1 &= 0 \\
  x_2 &= a \\
  x_3 &= b \\
\end{align*}

\[
  x = \begin{bmatrix}
    0 \\ 1 \\ 0
  \end{bmatrix} a 
  + \begin{bmatrix}
    0 \\ 0 \\ 1
  \end{bmatrix} b
  \text{ a plane in } \bbR^3
\]

Left null space $A^T y = 0$:

\[
  \begin{bmatrix}
    1 & 0 \\ 0 & 0 \\ 0 & 0
  \end{bmatrix}
  \begin{bmatrix}
    y_1 \\ y_2
  \end{bmatrix} = \vec{0}
\]

Dimension of row space corresponds with \# of linearly independent rows. $C(A^T)$ dimension is $r$ ($r$ linearly independent rows). If we have $A_{m \times n}$, then $r \le m$ and $r \le n$. $C(A)$ dimension is $r$ (even though they live in different spaces).

The dimension of $N(A)$ is $n-r$. Null space lives in $\bbR^n$

The dimension of $N(A^T)$ is $m-r$.

%!TEX root = ../notes.tex
\section{October 11 Lecture}

Final projects: think about what brought you to study Applied Math. What problems do you like to solve? We'll find some linear algebra componenet to it.

Last class:

\paragraph{Row space of $A$} $A \rightarrow U$, the `$r$' non zero rows are a basis for the row space $C(A^T)$. It has dimension $r$ and it is a subspace of $\bbR$. The row space of $U$ is the same as the row space of $A$, since they only differ by linear combinations of rows.

The row space of $A$ and $U$ have the same basis.

\paragraph{Null space of $A$} $N(A) = N(U)$. If $r$ rows are linearly independent $\Rightarrow$ there are $(n-r)$ free variables the dimension of $N(A) = n-r$. 

Null space definition:

\[
  x \in R^n s.t. Ax = 0
\]

\paragraph{Column space of $A$} $C(A)$. When we transform $A$ to $U$, the first non-zero elements' index in each row determines which variable of $x$ will be a pivot variable, suggesting that those indices determine the column space of $A$. The column space of $A$ is \textit{not} the same column space of $U$. Dimension of $C(A) = r$. $C(A)$ is a subspace of $\bbR^m$.

\paragraph{Left null space}

\[
  \begin{bmatrix}
    y_1 & y_2 & \cdots & y_m
  \end{bmatrix}
  \begin{bmatrix}
    \\ A^T \\ \\
  \end{bmatrix}_{m \times n} =
  \underbrace{\begin{bmatrix}
    0 & \cdots & 0
  \end{bmatrix}}_n
  \text{ it is a subspace of } \bbR^m
\]

$N(A^T)$ is $A^T y = 0$. If $y$ is in the null space of $A^T$, then $y^T$ is in the left null space of $A$ ($y^TA = 0$).

\subsection{Fundamental Theorem of Linear Algebra}

\begin{itemize}
  \item $\dim(C(A)) = r$
  \item $\dim(C(A^T)) = r$
  \item $\dim(N(A)) = n-r$
  \item $\dim(N(A^T)) = m-r$
\end{itemize}

$Ax=b$. Case: $(m \le n)$

\[
  \begin{bmatrix}
    & & & & & & \\
    & & & A & & & \\
    & & & & & & 
  \end{bmatrix}_{m \times n}
  \begin{bmatrix}
    \\ \\ \\ x \\ \\ \\ \\
  \end{bmatrix}_{n \times 1} = 
  \begin{bmatrix}
    \\ \\ b \\ \\ \\
  \end{bmatrix}_{m \times 1}
\]

\paragraph{Existance:} If $A$ has the maximum number of linearly independent rows($=m$) $A$ is said to have full ``row'' rank. There exists at least one solution for any $b$

In this case, $A$ has a right inverse. An example:

\begin{align*}
  A &= 
  \begin{bmatrix}
    4 & 0 & 0 \\ 0 & 5 & 0
  \end{bmatrix}
  C =
  \begin{bmatrix}
    1/4 & 0 \\ 0 & 1/5 \\ \alpha & \beta
  \end{bmatrix} \\
  AC &= \begin{bmatrix}
    1 & 0 \\ 0 & 1
  \end{bmatrix}
\end{align*}

In this example, the right inverse is not unique! Any values of $\alpha$ and $\beta$ will work.
  
\begin{align*}
  ACb &= b \\
  Ax &= b
  x &= Cb
\end{align*}

\paragraph{Uniqueness} $m \ge n$

\[
  \begin{bmatrix}
    & & \\
    & & \\
    & A & \\
    & & \\
    & & \\
  \end{bmatrix}_{m \times n}
  \begin{bmatrix}
    \\ x \\ \\
  \end{bmatrix}_{n \times 1} = 
  \begin{bmatrix}
    \\ \\ \\ \\ b \\ \\ \\ \\ \\
  \end{bmatrix}_{m \times 1}
\]

If all columns of $A$ are linearly independent, $A$ is said to be full ``column'' rank.

If $b \in C(A) \exists$ 1 unique solution. If $b \!in C(A)$, then no solution.

$\bbR^n$: normed vector space (with an innter product)

\[
  \<x,y\> = \sum_{i=1}^n x_i y_i
\]
\[
  \norm{x}^2 = \<x,x\> = x^T x
\]

Interesting property:

\[
  \<x,y\> = \norm{x}\norm{y}\cos \theta
\]

$x$ and $y$ are said to be orthogonal if $\<x,y\> = 0$. If we have $k$ non-zero vectors ($v_1,\cdots, v_k$) are mutually orthogonal, they are linearly independent. Then we can say that there is only one combination that satisfies the following:

\[
  c_1v_1 + c_2v_2 + \cdots + c_kv_k=0
\]

All $c_i$ must be 0. Proof:

\begin{align*}
  v_1^T(c_1v_1 + c_2v_2 + \cdots + c_kv_k)&=0 \\
  v_1^Tc_1v_1 + v_1^Tc_2v_2 + \cdots + v_1^Tc_kv_k &=0 \\
  c_1 \norm{v_1}^2 &= 0
\end{align*}

Repeat for all $v_i \Rightarrow c_i = 0 \forall i \Rightarrow \{v_k\}$ are linearly independent.

\[
  e_i = \begin{bmatrix}
    0 \\ \vdots \\ 1 \\ \vdots \\ 0
  \end{bmatrix} \text{ 1 at the } i\text{th element}
\]

$\{e_i\}$ is an orthonormal basis of $\bbR^n$. Any vector in $\bbR^n$ can be generated as a linear combination of them. If every vector in a subspace $V$ is orthogonal to every vector in subspace $W \Rightarrow V$ and $W$ are said to be orthogonal subspaces.

\begin{enumerate}
  \item The row space of $A$, $C(A^T)$ is the orthogonal complement to $N(A)$.
  \item The column space of $A$, $C(A)$ is the orthogonal complement to the left null space $N(A^T)$.
\end{enumerate}

\begin{enumerate}
  \item Why is this true? By definition, the null space are vectors $x$ such that $Ax = 0$. If $A$ is $m \times n$. This means that the inner product of every row of $A$ with $x$ must be 0. In other words, $x$ is orthogonal to every row of $A$. The rows define the row space $C(A^T)$, so (1) is true.
  \item The left null space is defined as $y$ such that $y^TA = 0$. If $A$ is $m \times n$, then the inner product of $y$ and each of the $n$ columns of $A$ must be 0.
\end{enumerate}

%!TEX root = ../notes.tex
\section{October 16 Lecture}

\paragraph{Fundamental theorem of Linear Algebra}
\begin{itemize}
  \item The null space of $A$, $N(A)$, is the orthogonal complement of the row space of $A$ (living in $\bbR^n$)
  \item The left null space $N(A^T)$ is the orthogonal complement of the column space of $A$
\end{itemize}

\[
  \min_{\hat{x} \in \bbR} \norm{a \hat{x} - b}
\]

\begin{align*}
  b-\hat{x}a &\perp a \\
  \<a,b-\hat{x}a\> &= 0\\
  a^t(b-\hat{x}a) &= 0\\
  \hat{x} &= {a^t b \over a^t a} \\
  p &= \hat{x} a
\end{align*}

In $\bbR^n$:

\[
  e_i = \begin{bmatrix}
    0 \\ \vdots \\ 1 \\ \vdots \\ 0
  \end{bmatrix} \text{ 1 at the } i\text{th element}
\]

These $\{e_k\}$ form an orthonormal basis for $\bbR^n$.

\[
  \bbR^5 : v = \begin{bmatrix}
    1 \\ 2 \\ 3 \\ 4 \\ 5
  \end{bmatrix} = 
  1 \begin{bmatrix} 1 \\ 0 \\ 0 \\ 0 \\ 0  \end{bmatrix} +
  2 \begin{bmatrix} 0 \\ 1 \\ 0 \\ 0 \\ 0 \end{bmatrix} +
  3 \begin{bmatrix} 0 \\ 0 \\ 1 \\ 0 \\ 0 \end{bmatrix} +
  4 \begin{bmatrix} 0 \\ 0 \\ 0 \\ 1 \\ 0 \end{bmatrix} +
  5 \begin{bmatrix} 0 \\ 0 \\ 0 \\ 0 \\ 1 \end{bmatrix}
\]

\[
  v = c_1 e_1 + ... + c_5 e_5 \text{ where } c_i = {\< v, e_i \> \over \norm{e_i}^2}
\]

\subsection{Fourier series}

Similarly, we can express an arbitrary function $f(x)$ as a sum:

\[
  f(x) = a_0 + \sum_{k=1}^\infty a_k \cos{kx} + b_k \sin{kx}
\]

The basis functions are $\sin(kx)$ and $\cos(kx)$.

Define $f(x) : [-\pi,\pi] \to \bbR$.

\[
  \int_{-\pi}^\pi \abs{f(x)}^2 < M
\]

$M$ is finite. If $f,g \in L^2([-\pi,\pi]) \Rightarrow \<f,g\> = \int_{-\pi}^\pi f(x)g(x)dx$

Functions that map from finite intervals to the real numbers form a vector space.

What should $a_k, b_k$ and $a_o$ be?

\[
  a_k = {1 \over \pi} \int_{-\pi}^\pi f(x)\cos(kx)dx
\]

\[
  b_k = {1 \over \pi} \int_{-\pi}^\pi f(x)\sin(kx)dx
\]

% \[
%   a_0 = {1 \over \pi} \int_{-\pi}^\pi f(x)dx
% \]

Euler's formula:

\[
  e^{ikx} = \cos{kx} + i \sin{kx}
\]

\[
  f(x) = \sum_{k=-\infty}^\infty c_k e^{ikx} \text{ with } c_k = {1 \over 2\pi} \int_{-\pi}^\pi f(x) e^{ikx} dx
\]

How do we bring this problem to finite dimension, so we can solve a problem of the form $Ax=b$?

Suppose you have a discrete signal on a fixed interval.

\begin{align*}
  f(x) &= c_0 + c_1 e^ikx + c_2 e^i2kx + c_3 e^i3kx \\
  f(0) &= f_0 = c_0 + c_1 + c_2 + c_3  \\
  f(\pi/2) &= f_1 = c_0 + c_1 i - c_2 - c_3 i \\
  f(\pi) &= f_2 \\
  f(3\pi/2) &= f_3
\end{align*}

This gives us matrix $A$:

\[
  \begin{bmatrix}
    1 & 1 & 1 & 1 \\
    1 & i & i^2 & i^3 \\
    1 & i^2 & i^4 & i^6 \\
    1 & i^3 & i^6 & i^9 \\    
  \end{bmatrix}
\]

For use in problem $Ac=f$:

\[
  c = \begin{bmatrix}
    c_0 \\ c_1 \\ c_2 \\ c_3
  \end{bmatrix},
  f = \begin{bmatrix}
    f_0 \\ f_1 \\ f_2 \\ f_3
  \end{bmatrix}
\]

Sunspots and crazy stuff.


\end{document}
