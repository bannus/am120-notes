%!TEX root = ../notes.tex
\section{September 25 Lecture}
Need to solve $Ax=b$. In theory, we can find $A^{-1}$. In practice, this is not necessary -- it takes too many operations!

\[
  A \begin{bmatrix}
    | & | & \cdots & |  \\
    x_1 & x_2 & \cdots & x_n  \\
    | & | & \cdots & |  \\
  \end{bmatrix} = 
  \begin{bmatrix}
    | & | & \cdots & |  \\
    e_1 & e_2 & \cdots & e_n  \\
    | & | & \cdots & |  \\
  \end{bmatrix} = 
\]

\begin{align*}
  A A^{-1} &= I \\
  x &= A^{-1}b
\end{align*}

In practice, we find $LU$. Then we solve $Lc = b$L

\begin{align*}
  Lc &= b \\
  c = L^{-1}b
\end{align*}

Then, we solve $Ux = c$:

\begin{align*}
  Ux &= c
  L^{-1} A x &= L^{-1} b
\end{align*}

This algorithm fails if any of the pivots are 0.

\paragraph{Singular matrix} A matrix is singular if $\det(A) \not= 0 \iff Ax = b$ has a unique solution.

\subsection{Floating point numbers}

\[
  fl(1 + \epsilon) > 1 \Rightarrow \epsilon = {1 \over 2} \beta^{1 - \rho}
\]

Machine precision ($\epsilon_{\text{mach}}$): single ~$10^{-8}$; double ~$10^{-16}$.

\begin{align*}
  a &= 1.23456 \times 10^2
  a &= -1.23455 \times 10^{-2}
  a &= -1.11123 \times 10^{-3}
  a+b+c &= (1 \times 10^{-3}) -1.11123 \times 10^{-3} = -1.1123 \times 10^{-4}
\end{align*}

This is the result if we calculate $(a+b) + c$. Notice that if we use 6-digit arithmetic and calculate $a+(b+c)$, we get 0! Floating point arithmetic is not associative.

\[
  \sum_{n=1}^\infty {1 \over n} \to
\]

\begin{enumerate}
  \item ${1 \over n}$ < UFL
  \item $\sum_{n=1}^{k_2} {1 \over n}$ > OFL
  \item fl$\left(\left(\sum_{n=1}^{k_3} {1 \over n}\right) + {1 \over k_3 + 1} \right) = \sum_{n=1}^{k_3} {1 \over n}$
\end{enumerate}

\subsection{Catastrophic cancellation}

If we add a very small number to a big number, sometimes we get the same big number! Like $1.0 \times 10^8 + 1.0 \times 10^{-9} = 1.0 \times 10^8$.

\begin{align*}
  A + b &= A \\
  A - b &= A
\end{align*}

Partial pivoting minimized catastrophic cancellation. *Bunch of matrices about the steps of partial pivoting, that I don't even think were accurate*

\paragraph{Theorem} For a non-singular and square matrix $A$, $\exists P$ (permutation matrix) thta reorders rows of $A$ to avoid zeros in the pivot positions. $Ax=b$ has a unique solution and with the rows ordered ``in advance.''

\[
  PA = LU \text{ and } L \text{ and } U \text{ are unique}
\]