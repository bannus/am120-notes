%!TEX root = ../notes.tex
\section{October 9 Lecture}

The transpose of a matrix $A$ (denoted $A^T$) is a matrix with columns directly from rows of $A$ (the $i$th row becomes the $i$th column of $A^T$). $(AB)^T = B^T A^T$.

\paragraph{Colummn space of $A$} Denoted $C(A)$. Contains all linear combinations of the columns of $A$. $C(A)$ is a subspace of $\bbR^m$.

\paragraph{Complete solution} to the problem $Ax=b$ can be expressed as $x=x_p + x_n$. $x_n \in N(A), x_p$ a particular solution.

\begin{align*}
  A &= \begin{bmatrix}
    1 & 2 \\ 2 & 4
  \end{bmatrix} \\
  A x_p &= \begin{bmatrix}
    3 \\ 6
  \end{bmatrix}
\end{align*}

Should know
\begin{enumerate}
  \item What is a vector space?
  \item When a set of vectors are linearly independent?
  \item Dimension of a vector space
  \item Basis for a vector space
\end{enumerate}

\paragraph{Theorem} $N(A)$ contains only the vector iff zero the columns of $A$ are linearly independant. Therefore, if $A$ is square, $\exists! x $for any$ b$.

\[
  A = \begin{bmatrix}
    1 & 3 & 3 & 2 \\
    2 & 6 & 9 & 7 \\
    -1 & -3 & 3 & 4
  \end{bmatrix} =
  \begin{bmatrix}
    1 & 0 & 0 \\
    2 & 1 & 0 \\
    -1 & 2 & 1 
  \end{bmatrix}_L
  \begin{bmatrix}
    1 & 3 & 3 & 2 \\
    0 & 0 & 3 & 3 \\
    0 & 0 & 0 & 0
  \end{bmatrix}_U
\]

The \textit{row space of $A$} is the column space of $A^T$ ($C(A^T)$). It is a subspace of $\bbR^n$.

The left null space of $A$ is the null space of $A^T$. $y \in N(A^T)$ if $A^Ty=0$ iff $y^T A = 0$.

Let's calculate the null space of $A$. We've broken $A$ into $LU$. Since $U$ was obtained by adding and subtracting rows of $A$, it follows that $N(A) = N(U)$.

\[
  \begin{bmatrix}
    1 & 3 & 3 & 2 \\
    0 & 0 & 3 & 3 \\
    0 & 0 & 0 & 0
  \end{bmatrix}_U 
  \begin{bmatrix}
    u \\ v \\ w \\y
  \end{bmatrix} = 
  \begin{bmatrix}
    0 \\ 0 \\0
  \end{bmatrix}
\]

\begin{align*}
  3w + 3y &= 0 \Rightarrow w = -y \\
  u + 3v - 3y + 2y &= 0 \\
  u &= y - 3v
\end{align*}

\[
  x \in N(A) = \begin{bmatrix}
    y - 3v \\ v \\ -y \\ y
  \end{bmatrix} = 
  y \begin{bmatrix}
    1 \\ 0 \\ -1 \\ 1
  \end{bmatrix} + 
  v \begin{bmatrix}
    -3 \\ 1 \\ 0 \\ 0
  \end{bmatrix}
\]

The linear combination of these two vectors generates a plane in $\bbR^4$, the null space of $A$.

\[
  Ax = \begin{bmatrix}
    b_1 \\ b_2 \\ b_3
  \end{bmatrix}
\]

\[
  \begin{bmatrix}
    1 & 3 & 3 & 2 \\
    0 & 0 & 3 & 3 \\
    0 & 0 & 0 & 0
  \end{bmatrix}_U 
  \begin{bmatrix}
    u \\ v \\ w \\y
  \end{bmatrix} = 
  \begin{bmatrix}
    b_1 \\ b_2 - 2b_1 \\ (b_3 + b_1) - 2(b_2 - 2b_1)
  \end{bmatrix} = 
    \begin{bmatrix}
    b_1 \\ b_2 - 2b_1 \\ b_3 - 2b_2 + 52b_1
  \end{bmatrix}
\]

\[
  \iff b_3 - 2b_2 + 5b_1 = 0 \text{ solvability condition}
\]

Now, we choose:

\[
  b = \begin{bmatrix}
    1 \\ 5 \\ 5
  \end{bmatrix},
  \tilde{b} = \begin{bmatrix}
    1 \\ 3 \\ 0
  \end{bmatrix}
\]

\[
  \begin{bmatrix}
    1 & 3 & 3 & 2 \\
    0 & 0 & 3 & 3 \\
    0 & 0 & 0 & 0
  \end{bmatrix}_U 
  \begin{bmatrix}
    u \\ v \\ w \\y
  \end{bmatrix} =
  \begin{bmatrix}
    1 \\ 3 \\ 0
  \end{bmatrix}
\]

\begin{align*}
  3w + 3y &= 3 \Rightarrow w = 1 - y \\
  u + 3v + 3(1-y) + 2y &= 1 \Rightarrow u = -2 - 3v + y
\end{align*}

Now we can write

\[
  \begin{bmatrix}
    1 & 3 & 3 & 2 \\
    0 & 0 & 3 & 3 \\
    0 & 0 & 0 & 0
  \end{bmatrix}_U 
  \begin{bmatrix}
    u \\ v \\ w \\y
  \end{bmatrix} =
  \begin{bmatrix}
    1 \\ 3 \\ 0
  \end{bmatrix} \Rightarrow
  x = \underbrace{\begin{bmatrix}
    -2 \\ 0 \\ 1 \\ 0
  \end{bmatrix}}_{x_p}+
  \underbrace{v \begin{bmatrix}
    -3 \\ 1 \\ 0 \\0
  \end{bmatrix} + 
  y \begin{bmatrix}
    1 \\ 0 \\ -1 \\ 1
  \end{bmatrix}}_{x_n}
\]

Remarks:
\begin{itemize}
  \item The null space of $A$, $N(A)$, and the row space of $A$, $C(A^T),$ are subspaces of $\bbR^n$.
  \item The left null space, $N(A^T)$, and column space of $A$, $C(A)$ are subspace of $\bbR^m$.
\end{itemize}

Transform $A \underbrace{\rightarrow}_\text{using Gauss elim} U$ we can immediately identify a basis for $C(A^T)$.

\[
  A = \begin{bmatrix}
    1 & 0 & 0 \\ 0 & 0 & 0
  \end{bmatrix},
  A^T = \begin{bmatrix}
    1 & 0 \\ 0 & 0 \\ 0 & 0
  \end{bmatrix}
\]
\[
  C(A) = \begin{bmatrix}
    1 \\ 0
  \end{bmatrix}d \text{ line in } \bbR^2
\]

Row space of $A$ = $\begin{bmatrix} 1 \\ 0 \\ 0 \end{bmatrix}$ c lives in $\bbR^3$

$N(A)$ lives in $\bbR^3: Ax = 0$:

\[
  A = \begin{bmatrix}
    1 & 0 & 0 \\ 0 & 0 & 0
  \end{bmatrix}
  \begin{bmatrix}
    x_1 \\ x_2 \\ x_3
  \end{bmatrix} = \vec{0}
\]

We see that $x_1$ must be 0, but the other values are free:

\begin{align*}
  x_1 &= 0 \\
  x_2 &= a \\
  x_3 &= b \\
\end{align*}

\[
  x = \begin{bmatrix}
    0 \\ 1 \\ 0
  \end{bmatrix} a 
  + \begin{bmatrix}
    0 \\ 0 \\ 1
  \end{bmatrix} b
  \text{ a plane in } \bbR^3
\]

Left null space $A^T y = 0$:

\[
  \begin{bmatrix}
    1 & 0 \\ 0 & 0 \\ 0 & 0
  \end{bmatrix}
  \begin{bmatrix}
    y_1 \\ y_2
  \end{bmatrix} = \vec{0}
\]

Dimension of row space corresponds with \# of linearly independent rows. $C(A^T)$ dimension is $r$ ($r$ linearly independent rows). If we have $A_{m \times n}$, then $r \le m$ and $r \le n$. $C(A)$ dimension is $r$ (even though they live in different spaces).

The dimension of $N(A)$ is $n-r$. Null space lives in $\bbR^n$

The dimension of $N(A^T)$ is $m-r$.
