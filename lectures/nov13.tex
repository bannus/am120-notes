%!TEX root = ../notes.tex
\section{November 13 Lecture}

\subsection{Quiz Questions}

\begin{itemize}
  \item Actual midterm will only differ from practice midterm by changing a few values\footnote{Hi Jimmy.}
  \item No SVD
  \item No QR within the context of finding eigenvalues/eigenvectors.
  \item No need to know IEEE standards for machine precision, max values
  \item Exam Location: maybe not here
  \item No calculator needed
  \item $n \times n$ matrix with $n$ non-zero eigenvalues has rank $n$
  \item Pseudoinverse was not yet discussed
\end{itemize}

\subsection{Spectral Theorem}
Every real symmetric ($A^T=A$) matrix $A$ can be diagonalized by an orthogonal matrix $Q$. (The eigenvalues of $A$ are real in this case). $Q$ contains the orthonormal eigenvalues of $A$.

\[
  A = Q \Lambda Q^T = \sum_{i=1}^n \lambda_i x_i x_i^T
\]

\subsection{Singular Value decomposition}
Any $m \times n$ matrix $A$ can be factored into $A = U \Sigma V^T$ = (orthogonal)(``diagonal'')(orthogonal)

\begin{align*}
  A A^T (m \times m) &= (U \Sigma V^T) (V \Sigma^T U^T) = U \Sigma \Sigma^T U^T \\
  A^T A (n \times n) &= (V \Sigma^T U^T) (U \Sigma V^T) = V \Sigma^T \Sigma V^T
\end{align*}

In the first line, $U$ contains the eigenvectors of $AA^T$ in its columns, and in the second line $V$ contains the eigenvectors of $A^TA$ in its columns. $\Sigma \Sigma^T$ is the eigenvalue matrix for $AA^T$, size $m \times m$ with $\set{r_i^2}$ on the diagonal.

$A A^T$ is a covariance matrix. If one finds the eigenvector that corresponds with the largest eigenvalue, we've found the direction of the largest variance (?).

\begin{align*}
  A v &= U \Sigma \Rightarrow
  A
  \begin{bmatrix}
    | & & | \\
    v_1 & \cdots & v_n \\
    | & & | \\
  \end{bmatrix}
  =
  \begin{bmatrix}
    | & & | \\
    u_1 & \cdots & u_m \\
    | & & | \\
  \end{bmatrix}
  \begin{bmatrix}
    \sigma_1 \\
    &\sigma_2 \\
    0& &\sigma_r \\
    & & & 0 \\
    & & & & \ddots
  \end{bmatrix}
\end{align*}

\paragraph{Theorem} $U$ and $V$ produce an orthonormal basis for all four fundamental spaces.
\begin{enumerate}
  \item First $r$ columns of $U$: column space of $A$
  \item First $m-r$ columns of $U$: left null space of $A$
  \item First $r$ columns of $V$: row space of $A$
  \item Last $n-r$ columns of $U$: null space of $A$
\end{enumerate}

\section{Midterm}

\begin{itemize}
  \item $LU$ factorization (multipliers $\rightarrow L$ with (-) sign)
  \item $LU \rightarrow$ solve $Ax = b$
  \item Finite precision (machine precision)
  \begin{itemize}
    \item relevance $\Rightarrow$ to what extent can we identify matrices that are singular? Gaussian elimination $\Rightarrow$ how ill-conditioned
  \end{itemize}
  \item Computer 4 fundamental spaces
  \item Particular solution based on if on column space
  \item Least squares using $QR$, normal equations
  \item Fundamental theorem of linear algebra
  \item Householder
  \item Computing eigenvectors and eigenvalues
  \item How to solve least squares solution using calculus (setting derivative equal to 0) or some other appraoch involving something perpendicular to something else (??)
\end{itemize}