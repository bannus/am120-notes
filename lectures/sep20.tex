%!TEX root = ../notes.tex
\section{September 20 Lecture}
First part of AM120 is to solve $Ax=b$ for arbitrary $\norm{A} = n$.

\[
	u_{11} = a_{11}, u_{22} = a_{22} 
\]

Pseudocode did not have 0s in $L$ and $U$. Second part of code is given $L$ and $b$, should output $c$. Third part takes $U$ and $c$ and outputs $x$.

Assignment 2 Due on Monday morning (9am).

This Doolittle algorithm can fail:
\begin{enumerate}
	\item If there is a pivot = 0
	\begin{enumerate}
		\item System is singular $\Rightarrow \det(A) = 0$. This means there is no solution or infinitely many solutions.
		\item We can exchange rows and `cure' system.
		\[
			\det(A) = \det(L) \det(U) = 1 \prod_{k=1}^n u_{kk}
		\]
	\end{enumerate}
\end{enumerate}


\paragraph{Example} From last class:

\[
	\begin{bmatrix}
		0.0001 & 1 \\ 1 & 1
	\end{bmatrix}
	\begin{bmatrix}
		x_1 \\ x_2
	\end{bmatrix} = 
	\begin{bmatrix}
		1 \\ 2
	\end{bmatrix}
\]

This had true solution:

\[
		x_1 = {10000 \over 9999}, x_2 = {9998 \over 9999}
\]

But with limited precision (three digit arithmetic), we got'

\[
		x_1 = 0, x_2 = 1
\]

What if we switch the rows?

\[
	\begin{bmatrix}
		1 & 1 \\ 0.0001 & 1
	\end{bmatrix}
	\begin{bmatrix}
		x_1 \\ x_2
	\end{bmatrix} = 
	\begin{bmatrix}
		2 \\ 1
	\end{bmatrix}
\]

\[
	\begin{bmatrix}
		1 & 0 \\ 10^{-4} & 1
	\end{bmatrix}_L
	\begin{bmatrix}
		1 & 1 \\ 0 & 1
	\end{bmatrix}
	\begin{bmatrix}
		x_1 \\ x_2
	\end{bmatrix} =
	\begin{bmatrix}
		2 \\ 1 - 2 \cdot 10^{-4}t
	\end{bmatrix}
\]


\subsection{Finite precision}

The computer represents a floating point number with a sign, exponent, and digits for the value itself. When we are talking about $n$-digit arithmetic, we are referring to the number of digits storing the value.

\[
	\begin{matrix}
		U = \text{max exponent} \\
		L = \text{lowest exponent} \\
		P = \text{mantissa number of digits} \\
		\beta = \text{base} \\
	\end{matrix}
\]

For example, for $L=-1, U=1, p=2$ and $\beta = 10$'

\[
	\text{(sign)} d_0 . d_1 d_2 \dots d_p \times 10^{\text{exponent}}
\]

\[
	\begin{matrix}
		\text{largest} & 9.9 \times 10^1 = 99\\
		\text{smallest (non-zero)} & 1.0 \times 10^{-1} = 0.1
	\end{matrix}
\]

Examples of real values in floating point systems:

\[
	\begin{matrix}
		& \beta & P & L & U \\
		\text{IEEE} & 2 & 24 & -126 & 123 & \text{single} \\
					& 2 & 53 & -1022 & 1023 & \text{double}\\ 
		\text{HP}   & 10 & 12 & -499 & 499
	\end{matrix}
\]

What is the total number of floating point numbers?

\[
	2(\beta - 1) \beta^{p-1} (u-l+1) + 1
\]

Largest representable number:

\[
	(\beta - 1).(\beta - 1)\dots(\beta - 1) \cdot \beta^U
\]

Smallest number (absolute value):

\[
	\beta^L (\text{underflow})
\]

\paragraph{Machine precision} Note that the difference between the real number and the floating point number chosen depends on exponent.

\[
	\forall x \in R, \exists \text{fl}(x) = \hat{x} \text{ such that } \abs{x-\hat{x}} \leq \sum \abs{x}
\]

Note that this is not really true for all $x \in R$ -- only within a certain range. $\epsilon_{\text{mach}}$ is the largest number s.t. fl($1 + \epsilon_{\text{mach}}) > 1$.

Floating point numbers are not associative:

\[
	A+(B+C)) \not= (A+B)+C
\]