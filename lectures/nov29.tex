%!TEX root = ../notes.tex
\section{November 1 Lecture}

\subsection{Differential equation models}

\paragraph{Predator-prey model}

\begin{align*}
  {dx \over dt} &= x (\alpha - \beta y) \\
  {dy \over dt} &= -y (\gamma - \delta x)
\end{align*}

Stability of this system is studied using the Jacobian matrix

\paragraph{SIR Model}

\begin{align*}
  {dS \over dt} &= -\beta S I\\
  {dI \over dt} &= \beta SI - \gamma I
  {dR \over dt} &= \gamma I\\
\end{align*}

Where $\beta$ is the rate of infection and $\gamma$ is the rate of recovery.

\paragraph{1 dimensional}

\[
  \begin{matrix}
    {du \over dt} = au & & u(t=0) = u_0
  \end{matrix}
\]

If $u=\beta e^{at} \implies {du \over dt} = \beta a e^{at} = au$. Then $u(t=0) = \beta e^0 = \beta$. Solution is $u = u_0 e^{at}$.

\paragraph{Stability} If $a$ is positive, it grows exponentially. If $a=0$, then the function is constant. If $a$ is negative, it decays to 0. (Asymptotic analysis)

\paragraph{2 dimensional}

\[
  \begin{matrix}
    {dv \over dt} = 4v - 5w & v(t=0) = 8
    {dw \over dt} = 2v - 3w & w(t=0) = 5
  \end{matrix}
\]

\begin{align*}
  {d \over dt} \begin{bmatrix}
    v \\ w
  \end{bmatrix}
  &=
  \overbrace{
    \begin{bmatrix}
      4 & -5 \\
      2 & -3
    \end{bmatrix}
  }^A
  \begin{bmatrix}
    v \\ w
  \end{bmatrix} \\
  {du \over dt} &= Au
\end{align*}

So we write:

\[
  \begin{matrix}
    v(t) = e^{\lambda t} y \\
    u(t) = e^{\lambda t} z
  \end{matrix}
   = x e^{\lambda t} \text{ for } x = 
  \begin{bmatrix}
    y \\ z
  \end{bmatrix}
\]

If this is true, then:

\begin{align*}
  {dv \over dt} &= y \cancel{e^{\lambda t}} = 4 \cancel{e^{\lambda t}}y - 5 \cancel{e^{\lambda t}} z \\
  {dw \over dt} &= z \cancel{e^{\lambda t}} = 2\cancel{e^{\lambda t}}y - 3 \cancel{e^{\lambda t}} z \\
\end{align*}

So we have:

\begin{align*}
  y \lambda &= 4y - 5z \\
  z \lambda &= 2y - 3z
\end{align*}

Equivalently:

\[
  Ax = \lambda x
\]

Eigenvalues and eigenvectors!

\[
  \begin{matrix}
    \lambda_1 = -1 & x_1 = \begin{bmatrix}
      1 \\ 1
    \end{bmatrix} \\
    \lambda_2 = 2 & x_2 = \begin{bmatrix}
      5 \\ 2
    \end{bmatrix}
  \end{matrix}
\]

\[
  \begin{bmatrix}
    v(t) \\ w(t)
  \end{bmatrix}
  =
  c_1 \begin{bmatrix}
    1 \\ 1
  \end{bmatrix}
  e^{-t}
  +
  c_2
  \begin{bmatrix}
    5 \\ 2
  \end{bmatrix}
  e^{2t}
\]

Now the question is do all solutions look like this? Yes. What is the value of $c_1$ and $c_2$? We evaluate at 0:

\[
  \begin{bmatrix}
    v(0) \\ w(0)
  \end{bmatrix}
  =
  \begin{bmatrix}
    8 \\ 5
  \end{bmatrix}
  =
  c_1 \begin{bmatrix}
    1 \\ 1
  \end{bmatrix}
  +
  c_2
  \begin{bmatrix}
    5 \\ 2
  \end{bmatrix}
\]

So we have:

\begin{align*}
  \begin{bmatrix}
    1 & 5 \\ 1 & 2
  \end{bmatrix}
  \begin{bmatrix}
    c_1 \\ c_2
  \end{bmatrix}
  &=
  \begin{bmatrix}
    8 \\ 5
  \end{bmatrix}\\
  \begin{bmatrix}
    v(t) \\ w(t)
  \end{bmatrix}
  &=
  \underbrace{
    \begin{bmatrix}
      1 & 5 \\ 1 & 2
    \end{bmatrix}
  }_S
  \begin{bmatrix}
    e^{-t} & 0 \\ 0 & e^{2t}
  \end{bmatrix}
  S^{-1}
  \begin{bmatrix}
    v(0) \\ w(0)
  \end{bmatrix} \\
  \begin{bmatrix}
    c_1 \\ c_2
  \end{bmatrix}
  &=
  S^{-1}
  \begin{bmatrix}
    v(0) \\ w(0)
  \end{bmatrix}
\end{align*}

\paragraph{Theorem} If $A$ can be diagonalized ($A = S \Lambda S^{-1}$), then the differential equation

\[
  \begin{matrix}
    {du \over dt} = Au & \text for & u(0) = u_0
  \end{matrix}
\]

has the solution $ u(t) = e^{At} u_0 $.

\begin{itemize}
  \item In 1-D: $e^x = 1 + x + {x^2 \over 2!} + {x^3 \over 3!} + \dots$
  \item In n-D: $e^At = I + At + {(At)^2 \over 2!} + {(At)^3 \over 3!} + \dots$
\end{itemize}

Then, if $A = S \Lambda S^{-1}$, $A^n = S \Lambda^n S^{-1}$:

\begin{align*}
  e^{At} &= I + S \Lambda S^{-1} t + S \Lambda^2 S^{-1} t^2 + \dots \\
  e^{At} &= S(I + \Lambda t + \Lambda^2 t^2 + \dots) S^{-1} = S e^{\Lambda t} S^{-1} \\
\end{align*}

Rewriting, the general solution to any linear solution

\[
  u(t) = e^{At} u_0 = S e^{\Lambda t} S^{-1} u_0
\]

\[
  =
  \underbrace{
    \begin{bmatrix}
      | & | & & |\\
      x_1 & x_2 & \cdots & x_n\\
      | & | & & | 
    \end{bmatrix}
  }_S
  \begin{bmatrix}
    e^{\lambda_1 t}
    &e^{\lambda_n t}
    & & \ddots
    & & & e^{\lambda_n t}
  \end{bmatrix}
  S^{-1} u_0
\]

Solve equation

\[
  a_n {d^n y \over d t^n} + a_{n-1} {d^{n-1} y \over d t^{n-1}} + \dots + a_0 y = 0
\]

With $n-1$ initial conditions.

Can be transformed into a system of linear equations of first order with change of variables:

\begin{align*}
  x_1(t) &= y(t) \\
  x_2(t) &= {dy \over dt} \\
  &\vdots\\
  x_n(t) &=  {d^{n-1}y \over dt^{n-1}} \\
\end{align*}

This results in:

\begin{align*}
  {d x_1 \over dt} &= x_2 (= {dy \over dt}) \\
  {d x_2 \over dt} &= x_3 \\
  &\vdots \\
  {d x_n \over dt} &= {d^n y \over dt^n} \\
\end{align*}

In other words:

\[
  d
  \begin{bmatrix}
    x_1 \\ x_2 \\ \vdots \\ x_n
  \end{bmatrix}
  =
  \begin{bmatrix}
    0 & 1 & 0 & 0 & \dots & 0\\
    0 & 1 & 1 & 0 & \dots & 0\\
    0 & 1 & 0 & 1 & \dots & 0\\
    \\
    a_0 \over a_n & a_1 \over a_n & \dots & & & a_{n-1} \over a_n & 
  \end{bmatrix}
  \begin{bmatrix}
    x_1 \\ x_2 \\ \vdots \\ x_n
  \end{bmatrix}
\]