\section{October 18 (Guest) Lecture}

J. Nathan Kutz from University of Washington.

If the determinant is 0 -- BAD.

Forearm shiver. This is how you win a wrestling match.

\begin{enumerate}
  \item $\det{A - \lambda I} \not= 0$. $\vec{X} = (A - \lambda I)^{-1} \cdot 0 = 0$
  \item $\det{A - \lambda I} = 0$
\end{enumerate}

\[
  A = \begin{bmatrix}
    1 & 3 \\ -1 & 5
  \end{bmatrix}
\]  

\[
  \begin{bmatrix}
    1 - \lambda & 3 \\ -1 & 5 - \lambda
  \end{bmatrix} \vec{x} = 0
\]

\begin{align*}
  (1 - \lambda)(5 - \lambda) + 3 &= 0
  \lambda^2 - 6 \lambda + 8 &= 0
  (\lambda - 2)(\lambda - 4) &= 0
  \lambda &= 2,4
\end{align*}

Protip: don't do algebra or spell in public.

\begin{align*}
  \lambda &= 2: \begin{bmatrix}
    -1 & 3 \\ -1 & 3
  \end{bmatrix} \vec{x} = 0\\
  \vec{x} &= \begin{bmatrix}
    x_1 \\ x_2
  \end{bmatrix} \\
  x_1 &= 3x_2
\end{align*}

One equation, two unknowns $\to$ infinite solutions.

\begin{align*}
  \lambda &= 4: \begin{bmatrix}
    -3 & 3 \\ -1 & 1
  \end{bmatrix} \vec{x} = 0\\
  -x_1 + x_2 &= 0
\end{align*}

Dot products! Geometrically, indicates how much one is projected onto the other. Help

\subsection{Face Recognition}

We can read images with the followig command:

\verb!A = imresize(double(rgb2gray(imread('pic','jpeg'))),[120 80])!

Then, we reshape them into a single row:

\verb!a = reshape(A,1,120*80)!

We use each of these rows to assemble an $n \times 120*80$ matrix $B$. Then, we calculate $C = B^T B$.

We use the command \verb![V,D] = eigs(C,20,'lm')! to get the first 20 largest magnitude values.

The eigenvectors define a basis of faces. Each face has a different (hopefully unique) set of coefficients that can be used for face recognition.
