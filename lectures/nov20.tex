%!TEX root = ../notes.tex
\section{November 20 Lecture}

\subsection{Markov matrices}

\paragraph{Example} Each year ${1 \over 10}$ of the people outside of California move in, and ${2 \over 10}$ of the people inside California move out. Start with $y_0$ people outside and $z_0$ people inside.

At the end of the first year


\[
  \begin{matrix}
    \begin{matrix}
      y_1 = 0.9 y_0 + 0.2 z_0 \\
    z_1 = 0.1 y_0 + 0.8 z_0
    \end{matrix}
    & \Rightarrow &
    \begin{bmatrix}
      y_1 \\ z_1 
    \end{bmatrix}
    =
    \begin{bmatrix}
      0.9 & 0.2 \\ 0.1 & 0.8 
    \end{bmatrix}
    \begin{bmatrix}
      y_0 \\ z_0
    \end{bmatrix}
  \end{matrix}
\]

Notes:
\begin{enumerate}
  \item Since the total number of people stays fixed, each column of this matrix adds up to 1
  \item The numbers outside and inside can never become negative. The matrix has nonnegative elements (and the powers of ``$A^k$'' are all nonnegative) and history is completely disregarded. Each $u_{k+1} = \begin{bmatrix}
    y_{k+1} \\ z_{k+1}
  \end{bmatrix}$ depends only on the current $u_k$.
\end{enumerate}

After $k$ years, $u_k = Au_{k-1} = A^2 u_{k-2} = \dots = A^k u_0$.

\begin{align*}
  A &= S \Lambda S^{-1} \\
  A^k &= S \Lambda^k S^{-1}
\end{align*}

For the california example:

\begin{align*}
  \det(A - \lambda I)           &= 0\\
  \lambda^2 - 1.7 \lambda + 0.7 &= 0\\
  \lambda_1 = 1,\,& \lambda_2 = 0.7\\ 
  A &=
  {1 \over 3}
  \begin{bmatrix}
    2 & 1 \\ 1 & -1
  \end{bmatrix}
  \begin{bmatrix}
    1 & 0 \\ 0 & 0.7
  \end{bmatrix}
  \begin{bmatrix}
    1 & 1 \\ 1 & -2
  \end{bmatrix}
\end{align*}

Skipping ahead some steps:

\begin{align*}
  u_k &= S \Lambda^k S^{-1}u_0 \\
      &= (y_0 + z_0) \lambda_1^k
      \begin{bmatrix}
        {2 / 3} \\ {1 / 3}
      \end{bmatrix} +
      (y_0 - 2z_0) \lambda-2^k
      \begin{bmatrix}
        {1 / 3} \\ -{1 / 3}
      \end{bmatrix}
\end{align*}

\paragraph{Remark} The steady state is the eigenvector of $A$ corresponding to $\lambda = 1$.

\paragraph{Definition} A ``Markov Matrix'' has all $a_{ij} \geq 0$ with each column adding up to 1.
\begin{enumerate}
  \renewcommand{\theenumi}{\roman{enumi}}
  \item $\lambda_1$ is an eigenvalue of $A$
  \item Its eigenvector $x_1$ is non-negative and it is a steady state since $Ax_1 = \lambda_1 x_1 = x_1$ 
  \item The other eignevalues satisfy $\abs{\lambda_i} \leq 1$

  Note: if $A$ or any power of $A$ has all positive entries $\implies$
\end{enumerate}

\paragraph{Example} Car rental company has three rental locations (!,2,3). A customer may rent a car from any location and return it to any of the three locations.

Questions: where should I place these stations? Once you have a set of locations, you'd like to know what is going to happen to the distribution of bikes.

\[
  A = 
  \left.
  \overbrace{
  \begin{bmatrix}
    0.8 & 0.3 & 0.2 \\
    0.1 & 0.2 & 0.6 \\
    0.1 & 0.5 & 0.2
  \end{bmatrix}}^{\text{Rented From (1,2,3)}}
  \right\} \text{Returned to location}
\]

Suppose I have a car in location 2. What is the probability after $n$ steps that it is at location 1, 2, or 3.

\begin{align*}
  x_0 &= \begin{bmatrix}
    0 \\ 1 \\0
  \end{bmatrix} \\
  x_1 &= A x_0 = 
  \begin{bmatrix}
    0.3 \\ 0.2 \\ 0.5
  \end{bmatrix}
\end{align*}

For $k \geq 11$:

\[
  x_k = A^k x_0 = 
  \begin{bmatrix}
    0.557 \\ 0.230 \\ 0.213
  \end{bmatrix}
\]