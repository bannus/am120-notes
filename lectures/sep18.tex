\section{September 18 Lecture}

To review: solving $Ax=b$:
\begin{enumerate}
	\item Find $LU=A$
	\item Solve for $c$ in $Lc=b$ (forward substitution)
	\item Solve for $x$ in $Ux=c$ (back-subst)
\end{enumerate}

Multipliers to find $U$ are entries of $L$.

What is the \# of operations needed to get $LU$ factorization?

\[
	\approx {n^3 - n \over 3}
\]

\paragraph{Forward substitution} Number of operations:

$(n-1) + (n-2) + ... (1) \approx O(n^2)$

Back substitution is similar process (also $O(n^2)$). Most time consuming place is step 1.

\paragraph{Algorithm Failure} This $Ax=b$:

\[
	\begin{bmatrix}
		0 & 1 \\ 1 & 0
	\end{bmatrix}
	\begin{bmatrix}
		x_1 \\ x_2
	\end{bmatrix} = 
	\begin{bmatrix}
		1 \\ 1
	\end{bmatrix}
\]

has solution $\begin{bmatrix}
	1 \\ 1
\end{bmatrix}$. However, our algorithm won't find the answer because it can't switch rows. If the algorithm fails we have two options:

\begin{enumerate}
	\item We need to rearrange rows
	\item No solution
	\item Infinitely many solutions
\end{enumerate}

Example of (2):

\[
	\begin{bmatrix}
		0 & 1 \\ 0 & 0
	\end{bmatrix}
	\begin{bmatrix}
		x_1 \\ x_2
	\end{bmatrix} = 
	\begin{bmatrix}
		1 \\ 1
	\end{bmatrix}
\]

Example of (3):

\[
	\begin{bmatrix}
		0 & 1 \\ 0 & 1
	\end{bmatrix}
	\begin{bmatrix}
		x_1 \\ x_2
	\end{bmatrix} = 
	\begin{bmatrix}
		1 \\ 1
	\end{bmatrix}
\]

\paragraph{Fact} $\det(A) = \det(LU) = \det(L)\det(U)$

\[
	\det(U) = \prod_{i=1}^n u_{ii}
\]

\paragraph{Example} Consider this:

\[
	\begin{bmatrix}
		0.0001 & 1 \\ 1 & 1
	\end{bmatrix}
	\begin{bmatrix}
		x_1 \\ x_2
	\end{bmatrix} = 
	\begin{bmatrix}
		1 \\ 2
	\end{bmatrix}
\]

\[
	\begin{bmatrix}
		0.0001 & 1 \\ 0 & -9999
	\end{bmatrix}
	\begin{bmatrix}
		x_1 \\ x_2
	\end{bmatrix} = 
	\begin{bmatrix}
		1 \\ -9998
	\end{bmatrix}
\]

\[
	\Rightarrow x_2 = {9998 \over 9999}
\]

\[
	0.0001 x_1 + {9998 \over 9999} = 1
\]

\[
	\Rightarrow x_1= {10000 \over 9999}
\]

If we do all of this with limited precision (say 3 digits), we do the following:

\[
	\begin{bmatrix}
		0.0001 & 1 \\ 0 & -10^4
	\end{bmatrix}
	\begin{bmatrix}
		x_1 \\ x_2
	\end{bmatrix} = 
	\begin{bmatrix}
		1 \\ -10^4
	\end{bmatrix}
\]

\[
	\Rightarrow x_2 = 1
\]

Then if we use the first equation, we get

\[
	\Rightarrow x_1 = 0
\]

This is called \textbf{catastrophic cancellation}.