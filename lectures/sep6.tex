\section{September 6 Lecture}

There are two main problems that we will learn how to handle in this class.

\begin{enumerate}
	\item Find $x \in R^n$ such that $Ax=b$. $A$ is $m$ by $n$ matrix, $b \in R^n$ vector
	\item Find $x$ and $\lambda$ such that $Ax = \lambda x$
\end{enumerate}

\paragraph{Example}

\begin{tabular}{rr}
	$x + 2y = 3$ \\
	$4x + 5y = 6$
\end{tabular}

$$ A =
\begin{bmatrix}
	1 & 2 \\
	4 & 5 
\end{bmatrix}
\begin{bmatrix}
	x \\ y
\end{bmatrix}
=
\begin{bmatrix}
	3 \\ 6
\end{bmatrix}
$$

There are 3 ways to solve:
\begin{enumerate}
	\item $\begin{bmatrix}
		1 & 2 & 3 \\
		4 & 5 & 6
	\end{bmatrix} \Rightarrow
	\begin{bmatrix}
		1 & 2 & 3 \\
		0 & -3 & -6
	\end{bmatrix}\Rightarrow
	y=2,x=-1$ 
	\item $A^{-1}=\frac{1}{\text{det}(A)}
	\begin{bmatrix}
		5 & -2 \\
		-4 & -1 
	\end{bmatrix}$
	$det A = -3$ \\
	$x=A^{-1}b \Rightarrow \frac{1}{-3}
	\begin{bmatrix}
		+3 \\ -6
	\end{bmatrix}$
	\item Kramer's rule
\end{enumerate}

\paragraph{Summary}
Topics covered in next 3 classes:
\begin{enumerate}
	\item Geometric interpretation of solving linear systems
	\item Matrix notation (LU factorization)
	\item Singular cases (no solution, multiple soln's)
	\item Efficient way to solve $Ax=b$ using computers
\end{enumerate}

\subsection{Geometric interpretation}

\paragraph{Example} Graphical method:

Row interpretation (plot lines on coordinate system):

\begin{tabular}{c}
	$2x - y = 1$ \\
	$x + y = 5$
\end{tabular}
Solution: $x=2,y=3$

Column interpretation:

$$ \begin{bmatrix}
	2 \\ 1
\end{bmatrix} x + 
\begin{bmatrix}
	-1 \\ 1
\end{bmatrix} y =
\begin{bmatrix}
	1 \\ 5
\end{bmatrix}$$

\paragraph{Example} 3 by 3 system:

Each row represents a plane:

\begin{tabular}{c}
	2u + v + w = 5 \\
	4u - 6v + 0 = -2 \\
	-2u + 7v + 2w = 9
\end{tabular}

Remember: inner product of vector with another vector equals 0 $\Rightarrow$ orthogonal.

Column interpretation:

\[
	\begin{bmatrix}
		2 \\ 4 \\ 2
	\end{bmatrix} u + 
	\begin{bmatrix}
		1 \\ -6 \\ 7
	\end{bmatrix} v + 
	\begin{bmatrix}
		1 \\ 0 \\ 2
	\end{bmatrix} w =
	\begin{bmatrix}
		5 \\ -2 \\ 9
	\end{bmatrix}
\]

\paragraph{Example} Overdetermined system:
\[
	\begin{bmatrix}
		1 & 1 \\
		2 & 3 \\
		3 & 4 \\
	\end{bmatrix}
	\begin{bmatrix}
		c \\ d
	\end{bmatrix} = 
	\begin{bmatrix}
		2 \\ 5 \\ 7
	\end{bmatrix}
\]

Solution: $c = 1, d = 1$

In 4 dimensions, the rows represent 3-spaces, which are `flat' relative to 4 dimensional space. If we intersect $(x,y,z,t=0)$ with $(x,y,z=0,t)$, two three spaces, we get $(x,y)$ plane.

\[
	a_1 u + a_2 v + a_3 w + a_4 z = b
\]

\[
	A = (a_1 | a_2 | a_3 | a_4)
\]

\subsection{Algorithmic approach}
Generalizing to $n$ by $n$. How to solve $Ax = b$ in a way that scales well? Gaussian elimination (row reduction).


\begin{tabular}{c}
	$2u + v + w = 5$ \\
	$4u - 6v + 0 = -2$ \\
	$-2u + 7v + 2w = 9$ \\\hline
\end{tabular}

$\Rightarrow$
\begin{tabular}{c}
	$2u + v + w = 5$ \\
	   $-8v - 2w = -12$ \\
         $8v + 3w = 14$ \\\hline
\end{tabular}

$\Rightarrow$
\begin{tabular}{c}
	$2u + v + w = 5$ \\
	   $-8v - 2w = -12$ \\
         $w = 2$ \\\hline
\end{tabular}

$\Rightarrow v = 1, u = 1$

We need a process that takes:

\[
	A = \begin{bmatrix}
		2 & 1 & 1 \\
		4 & -6 & 0 \\
		-2 & 7 & 2 \\
	\end{bmatrix}
\]
\[
	x = \begin{bmatrix}
		u \\ v \\ w
	\end{bmatrix}
\]
\[
	b = \begin{bmatrix}
		5 \\ -2 \\ 9
	\end{bmatrix}
\]

...this $Ax=b$ problem and transforms it to a $Ux = \hat{b}$ problem. We can get an upper triangular matrix, and obtain solution by back substitution.

\paragraph{Problems} One issue that could arise is if the bottom row is all 0s: infinitely many solutions.
