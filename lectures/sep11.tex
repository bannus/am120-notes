%!TEX root = ../notes.tex
\section{September 11 Lecture}

Last class:
\begin{itemize}
	\item Introduced first central problem of linear algebra: solving linear equations
	\item Studied column and row interpretation of linear systems
	\item Introduced Gaussian elimination
\end{itemize}

\paragraph{Example} (from previous class) Row/Column interpretation.

\begin{tabular}{c}
	2u + v + w = 5 \\
	4u - 6v + 0 = -2 \\
	-2u + 7v + 2w = 9
\end{tabular}

Row: Three planes intersecting. Column: linear combination of three vectors

\[
	A = \begin{bmatrix}
		2 & 1 & 1 \\
		4 & -6 & 0 \\
		-2 & 7 & 2 \\
	\end{bmatrix}
\]

We were trying to figure out how to transform matrix $A$ into an upper triangular matrix.

\[
	\begin{bmatrix}
		1 & 0 & 0 \\
		-2 & 1 & 0 \\
		0 & 0 & 1 \\
	\end{bmatrix}
	\begin{bmatrix}
		5 \\ -2 \\ 9
	\end{bmatrix}
	=
	\begin{bmatrix}
		5 \\ -12 \\ 9
	\end{bmatrix}
\]

\[
	\begin{bmatrix}
		1 & 0 & 0 \\
		-2 & 1 & 0 \\
		0 & 0 & 1 \\
	\end{bmatrix}
	\begin{bmatrix}
		2 & 1 & 1 \\
		4 & -6 & 0 \\
		-2 & 7 & 2 \\
	\end{bmatrix} = 
	\begin{bmatrix}
		2 & 1 & 1 \\
		0 & -8 & -2 \\
		-2 & 7 & 2 \\
	\end{bmatrix}
\]

\paragraph{Matrix operations} Addition is associative: $A+B+C = (A+B) + C = A+ (B+C)$

Multiplication: dimension $m \times n$ multiplied by $n \times p$ results in $m \times p$ matrix. $AB \neq BA$.

\[
	\begin{bmatrix}
		0 & 1 \\ 1 & 0\\
	\end{bmatrix}
	\begin{bmatrix}
		2 & 3 \\ 7 & 8 \\ 
	\end{bmatrix} =
	\begin{bmatrix}
		7 & 8 \\ 2 & 3
	\end{bmatrix}
\]

\[
	\begin{bmatrix}
		2 & 3 \\ 7 & 8 \\ 
	\end{bmatrix} 
	\begin{bmatrix}
		0 & 1 \\ 1 & 0\\
	\end{bmatrix}=
	\begin{bmatrix}
		3 & 2 \\ 8 & 7
	\end{bmatrix}
\]

Matrix multiplication:
\[
	\begin{bmatrix}
	a_{11} & a_{12} & a_{13} & \cdots & a_{1n} \\
	a_{21} & a_{22} & a_{23} & \cdots & a_{2n} \\
	\vdots & &  & \ddots &  \\
	a_{m1} & a_{m2} & a_{m3}& \cdots & a_{mn} \\
	\end{bmatrix}
	\begin{bmatrix}
		x_1 \\ x_2 \\ \vdots \\ x_n
	\end{bmatrix} = 
	\begin{bmatrix}
		\sum_{i=1}^n a_{1i}x_i \\
		\sum_{i=1}^n a_{2i}x_i \\ 
		\vdots \\
		\sum_{i=1}^n a_{ni}x_i\\
	\end{bmatrix}
\]

\[
	\begin{bmatrix}
		| & | &  & | \\
		a_1 & a_2 & \cdots & a_n \\
		| & | &  & | \\
	\end{bmatrix}
		\begin{bmatrix}
		x_1 \\ \vdots \\ x_n
	\end{bmatrix} = 
	a_1x_1+a_2x_2+\dots+a_nx_n
\]

\[
	\begin{bmatrix}
		| & | &  & | \\
		a_1 & a_2 & \cdots & a_n \\
		| & | &  & | \\
	\end{bmatrix}
	\begin{bmatrix}
		| &| \\
		b_1 & b_2 \\
		| & | \\
	\end{bmatrix} = 
	\begin{bmatrix}
		| & | \\
		Ab_1 & Ab_2 \\
		| & | \\
	\end{bmatrix} 
\]

\paragraph{Row reduction} In matrix form

\[
	A = \begin{bmatrix}
		2 & 1 & 1 \\
		4 & -6 & 0 \\
		-2 & 7 & 2 \\
	\end{bmatrix}
\]

\begin{enumerate}
	\item Subtract 2 times row 1 to row 2
	\[
		\begin{bmatrix}
			1 & 0 & 0 \\
			-2 & 1 & 0 \\
			0 & 0 & 1 \\
		\end{bmatrix}_{E_{21}}
		\begin{bmatrix}
			2 & 1 & 1 \\
			4 & -6 & 0 \\
			-2 & 7 & 2 \\
		\end{bmatrix}_A = 
		\begin{bmatrix}
			2 & 1 & 1 \\
			0 & -8 & -2 \\
			-2 & 7 & 2 \\
		\end{bmatrix}
	\]
	\item Subtract -1 times row 1 to row 3
	\[
		\begin{bmatrix}
			1 & 0 & 0 \\
			0 & 1 & 0 \\
			1 & 0 & 1 \\
		\end{bmatrix}_{E_{31}}
		E_{21}A = 
		\begin{bmatrix}
			2 & 1 & 1 \\
			0 & -8 & -2 \\
			0 & 8 & 3 \\
		\end{bmatrix}
	\]
	\item Subtract -1 times row 2 to row 3
	\[
		\begin{bmatrix}
			1 & 0 & 0 \\
			0 & 1 & 0 \\
			0 & 1 & 1 \\
		\end{bmatrix}_{E_{32}}
		E_{31}E_{21}A = 
		\begin{bmatrix}
			2 & 1 & 1 \\
			0 & -8 & -2 \\
			0 & 0 & 1 \\
		\end{bmatrix}
	\]
\end{enumerate}

Originally we wanted to solve $Ax=b$. Now we have:

\[
	E_{32}E_{31}E_{21}A = U
\]

where $U$ is an upper triangular matrix.

\[
	E_{32}E_{31}E_{21}Ax = Ux
\]

Let's let $E_{32}E_{31}E_{21} = L$. Then, we have

\[
	\begin{matrix}
		L^{-1}A=U \\
		A = LU \\
		Ux = C = E_{32}E_{31}E_{21}b
	\end{matrix}
\]

Now we can solve by back substitution.

\[
	L^{-1}=E_{32}E_{31}E_{21} = \begin{bmatrix}
		1 & 0 & 0 \\
		-2 & 1 & 0 \\
		-1 & 1 & 1 
	\end{bmatrix}
\]


Matrix inverse properties:

\[
	\begin{matrix}
		(AB)^{-1} = B^{-1}A^{-1} \\
		(A_1A_2 \dots A_n)^{-1} = A_n^{-1} \dots A_2^{-1}A_1^{-1}
	\end{matrix}
\]

So we have:

\[
	L = \begin{bmatrix}
		1 & 0 & 0 \\
		2 & 1 & 0 \\
		-1 & -1 & 1
	\end{bmatrix} = E_{21}^{-1}E_{31}^{-1}E_{32}^{-1}
\]

\paragraph{Row reduction matrices} A matrix that subtracts $l$ times row $j$ from row $i$ is such that it includes $-l$ in row $i$, column $j$.

\[
	\begin{matrix}
			A = \begin{bmatrix}
		1 & 0 & 0 \\
		2 & 1 & 0 \\
		-1 & -1 & 1
	\end{bmatrix}_L
	\begin{bmatrix}
		2 & 1 & 1 \\
		0 & -8 & -2 \\
		0 & 0 & 1 \\
	\end{bmatrix}_U
	\end{matrix}
\]

$L$ is lower triangular and $U$ is upper triangular.

\begin{enumerate}
	\item Compute LU factorization
	\item Solve for $c$ in $Lc = b$ (forward substitution)
	\item Solve for $x$ in $Ux = c$ (back substitution)
\end{enumerate}

We want to solve $Ax=b$. We factor to get $LUx =b$. First we find $c$ such that $Lc = b$

\subsection{General Example}
\[
	\begin{bmatrix}
		l_{11} & 0 & 0 \\
		l_{21} & l_{22} & 0 \\
		l_{31} & l_{32} & l_{33}
	\end{bmatrix}
	\begin{bmatrix}
		c_1 \\ c_2 \\ c_3
	\end{bmatrix} = 
	\begin{bmatrix}
		b_1 \\ b_2 \\ b_3
	\end{bmatrix} \Rightarrow
	\begin{matrix}
		c_1 = b_1 / l_{11} \\
		c_2 = b_2 - b_1 l_{21}/ l_{11} \\
		c_3 = b_3 - l_{31}b_1 - l_{32}(b_2 - b_1 l_{21})
	\end{matrix}
\]

\[
	\begin{bmatrix}
		u_{11} & u_{12} & u_{13} \\
		0 & u_{22} & u_{23} \\
		0 & 0 & u_{33}
	\end{bmatrix}
	\begin{bmatrix}
		x_1 \\ x_2 \\ x_3
	\end{bmatrix} = 
	\begin{bmatrix}
		c_1 \\ c_2 \\ c_3
	\end{bmatrix} \Rightarrow
	\begin{matrix}
		x_3 = c_3/u_{33} \\
		x_2 = \frac{1}{u_{22}}(c_2 - u_{23}c_3/u_{33})\\
		x_1 = \dots
	\end{matrix}
\]