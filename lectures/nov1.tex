%!TEX root = ../notes.tex
\section{November 1 Lecture}

Least squares problem
Find $\hat{x}$ such that $Ax \approx b$. A is an $m \times n$ matrix ($m \geq n$), $x$ is $n$-vector, $b$ is $m$-vector. In other words, $J(x) = \norm{Ax - b}^2 = \norm{r}^2$ (residual) is minimized. If $A$ is full rank (column) then $r$ must be perpendicular to $C(A)$.

\begin{align*}
  &\Rightarrow A^Tr = A^T(Ax-b) = 0\\
  &\Rightarrow A^TA\hat{x} = A^Tb &\text{Normal equations}\\
  &\Rightarrow \hat{x} = (A^T A)^{-1}A^T b &\text{are the coeff (theory)}
\end{align*}

$P = A \hat{x}$ is the projection of $b$ onto $C(A)$. If $P'$ is the projection matrix, then $A\hat{x} = P'x = \underbrace{A(A^TA)^{-1} A^T}_{P'} b$.

Since $C(A)$ is the orthogonal complement of $N(A^T)$ (left-null space) in $R^m$:

\begin{align*}
  b &= P'b+(I-P')b & P'b &\in C(A) \\
  & & (I-P)b &\in N(A^T)
\end{align*}

We want to transform $A$ into a matrix that has $R_{n \times n}$ at the top and $0_{m \times m-n}$ at the bottom. $Q$ is an orthogonal transformation if $Q^TQ = I \Rightarrow Q^T=Q^{-1}$.

For any vector $v \in \bbR^{m}$:

\[
  \norm{Qv}^2 = (Qv)^T Qv = v^T Q^T Q v = v^T v = \norm{v}^2
\]

Householder transformations

\[
  Ha = H \begin{bmatrix}
    a_1 \\ a_2 \\ \vdots \\ a_m
  \end{bmatrix}
  =
  \begin{bmatrix}
    \alpha \\ 0 \\ \vdots \\ 0
  \end{bmatrix}
\]

If we have $A$:

\begin{align*}
  A &=
  \begin{bmatrix}
    a_{11} & a_{12} & \cdots & a_{1n} \\
    a_{21} & a_{22} & \cdots & a_{2n} \\
    & \ddots & \\
    \\
    a_{m1} & & & a_{mn}
  \end{bmatrix} \\
  H_1 A &=
  \begin{bmatrix}
    \alpha & * & \cdots & * \\
    0 & a_{22} & \cdots & a_{2n} \\
    0& \ddots & \\
    \vdots\\
    0 & & & a_{mn}
  \end{bmatrix} \\
  H_2 H_1 A &=
  \begin{bmatrix}
    \alpha & * & \cdots & * \\
    0 & * & \cdots & * \\
    0 & 0 & \ddots & a_{3n}\\
    \vdots\\
    0 & 0 & & a_{mn}
  \end{bmatrix} \\
\end{align*}

How to find $H$?

\begin{align*}
  H &= I - 2 {v v^T \over \norm{v}^2} \text{ where } a - \alpha e_1 \\
  \alpha &= -sign{a_1}\norm{a}
\end{align*}

Simple example:

\begin{align*}
  a &= \begin{bmatrix}
    4 \\ 3
  \end{bmatrix} \\
  v &= \begin{bmatrix}
    4 \\ 3
  \end{bmatrix}
  +
  5 \begin{bmatrix}
    1 \\ 0
  \end{bmatrix}
  =
  \begin{bmatrix}
    9 \\ 3
  \end{bmatrix} \\
  Ha &= \left[
    \underbrace{
      \begin{bmatrix}
        1 & 0 \\ 0 & 1
      \end{bmatrix}
    }_I
    -
    2
    {
      \begin{bmatrix}
        9 \\ 3
      \end{bmatrix}
      \begin{bmatrix}
        9 & 3
      \end{bmatrix}
      \over
      90
    }
  \right]
  \begin{bmatrix}
    4 \\ 3
  \end{bmatrix}
  =
  \begin{bmatrix}
    4 \\ 3
  \end{bmatrix}
  -
  {2(45) \over 90}
  \begin{bmatrix}
    9 \\ 3
  \end{bmatrix} \\
  &= \begin{bmatrix}
    -5 \\ 0
  \end{bmatrix}
\end{align*}

So how does this help? We're looking to transform:

\begin{align*}
  A &\to \begin{bmatrix}
    R \\ 0
  \end{bmatrix} \\
  Q^T A &= \begin{bmatrix}
    R \\0
  \end{bmatrix} \Rightarrow A = Q \begin{bmatrix}
    R \\ 0
  \end{bmatrix} \\
  A &= \begin{bmatrix}
    \underbrace{Q_1}_{n} & \underbrace{Q_2}_{m-n}
  \end{bmatrix}_{m \times m}
  \begin{bmatrix}
    R \\ 0
  \end{bmatrix} =
  Q_1 R + Q_20 = Q_1 R
\end{align*}

\subsection{Eigenvectors}

$A$ is a square matrix.

Find $x$ and $\lambda$ s.t. $Ax = \lambda x$. The $x$ are called eigenvectors and $\lambda$ are eigenvalues.

Suppose $\exists n \set{x_i}$ and $\set{\lambda_i} \Rightarrow$ any $y \in R^n, y = \sum_{i=1}^n c_i x_i$.

\begin{align*}
  Ay = A\sum_{i=1}^n c_i x_i = c_1 \lambda_1 x_1 + c_2 \lambda_2 x_2 + \cdots +c_n \lambda_n x_n
\end{align*}

Suppose we have $S$:

\begin{align*}
  S &=
  \begin{bmatrix}
    | & | &  & | \\
    x_1 & x_2 & \cdots & x_n \\
    | & | &  & | \\
  \end{bmatrix} \\
  AS &=
  \begin{bmatrix}
    | & | &  & | \\
    Ax_1 & Ax_2 & \cdots & Ax_n \\
    | & | &  & | \\
  \end{bmatrix} = 
  \begin{bmatrix}
    | & | &  & | \\
    \lambda_1x_1 & \lambda_2x_2 & \cdots & \lambda_nx_n \\
    | & | &  & | \\
  \end{bmatrix} \\
  &= \begin{bmatrix}
    | & | &  & | \\
    x_1 & x_2 & \cdots & x_n \\
    | & | &  & | \\
  \end{bmatrix}
  \begin{bmatrix}
    \lambda_1 & & & 0 \\
    & \lambda_2 &\\
    & & \ddots\\
    0& & & \lambda_n
  \end{bmatrix}
\end{align*}

So we can write:

\begin{align*}
  AS = S \Lambda
  A = S \Lambda S^{-1}
\end{align*}

This is another factorization -- a diagonalization process. If $A$ is symmetric and positive, then $S$ and $S^{-1}$ are orthogonal.           

