\section{September 27 Lecture}

\paragraph{Last Class} Theory: For a non-singular and square matrix $A$, $\exists P$ (permutation matrix) that reorders rows of $A$ to avoid zeroes in the pivot positions

$Ax=b$ has a unique solution, and with the rows ordered ``in advance:''

\[
	P A = L U \text{ where } L \text{ and } U \text{ are unique.}
\]

Note: If $A$ is singular, no $P$can produce a full set of pivots and elimination fails.

Gaussian elimination with partial pivoting: if a pivot is zero, then $A$ is singular

\[
	A x = 
	\begin{bmatrix}
		0 & 4 & 1 \\
		1 & 1 & 3 \\
		2 & -2 & 1
	\end{bmatrix}
	\begin{bmatrix}
		x_1 \\ x_2 \\ x_3
	\end{bmatrix} = 
	\begin{bmatrix}
		9 \\ 6 \\ -1
	\end{bmatrix} = 
	b
\]

\[
	PA =
	\begin{bmatrix}
		2 & -2 & 1 \\
		0 & 4 & 1 \\
		1 & 1 & 3
	\end{bmatrix},
	P = 
	\begin{bmatrix}
		0 & 0 & 1 \\
		1 & 0 & 0 \\
		0 & 1 & 0
	\end{bmatrix}
\]

\[
	\tilde{A} = PA = LU = 
	\begin{bmatrix}
		1 & 0 & 0
		0 & 1 & 0
		1/2 & 1/2 & 1
	\end{bmatrix}_{\tilde{L}}
	\begin{bmatrix}
		2 & -2 & 1
		0 & 4 & 1
		0 & 0 & 6
	\end{bmatrix}_{\tilde{U}}
\]

% for k = 1:n-1
% 	find p such that |a(p,k)| > |a(i,k)| for k <= i <= n
%	if p != k then
%		interchange rows k and p (and record permutation)
%	if a_kk = 0 then STOP

\subsection{Ill-conditioned Matrices}

The presence of round-off error makes it difficult to identify singular matrices.

\[
	A =
	\begin{bmatrix}
		1000 & 999 \\ 999 & 998		
	\end{bmatrix}	
	\rightarrow
	\begin{bmatrix}
		1 & 0 \\ .999 & 1
	\end{bmatrix}_L
	\begin{bmatrix}
		1000 & 999 \\
		0 & -.001
	\end{bmatrix}_U
\]


\[
	Ax = b = \begin{bmatrix} 1999 \\ 1997 \end{bmatrix}
	\rightarrow
	x = \begin{bmatrix} 1 \\ 1 \end{bmatrix}
\]

With limited precision (5-digit arithmetic), $A$ appears singular, because 0.999(999)=998.00.

\[
	Ax = \hat{b} = \begin{bmatrix} 1998.99 \\ 1997.01 \end{bmatrix} =
	b + \delta b = b + 10^{-2} \begin{bmatrix} -1 \\ 1 \end{bmatrix}
	\Rightarrow
	x = \begin{bmatrix} 20.97 \\ -18.99 \end{bmatrix}
\]

Small change in $b$ and same $A$, $x$ changes a lot. We call $A$ `ill-conditioned,' meaning that its `condition number,' $k(A)$ is big. This is the definition of condition number:

\[
	{\norm{\delta x} \over \norm{x}} \leq k(A) {\norm{\delta b} \over \norm{b}}
\]

Hilbert matrices are of this kind: changing $b$ a little bit, $x$ changes a lot (they are ill conditioned). Condition number of a singular matrix $A$ is $\infty$.

If ${\norm{\delta x} \over \norm{x}} > 1$ we don't expect to find a solution close to the one we were looking for.

In numberical calculations, singular matrices are indistinguishable from ill-conditioned matrix.


\[
	\begin{matrix}
		-{d^2 u \over dx^2} = f(x), 0 \leq x \leq 1 \\
		u(0) = c_1
		u(1) = c_2
	\end{matrix}
\]

\[
	u(x+h) = u(x) + h u'(x) + h^2 {u''(x) \over 2} + ... + h^k {d^k u \over d x^k}
\]

Discretize interval into points $x_i$, solve for value of $u$ at each point. At each point we can solve the problem as a linear algebra problem. We ignore higher terms, and say:

\[
	u'(x_i) \approx {u(x_i+h) - u(x_i) \over h}
\]

\[
	-u''(x_i) \approx -{u(x_{i+1} - 2 u(x_i) + u(x_{i-1})) \over h^2 } = f(x_i)
\]

This results in a large matrix:

\[
	\begin{bmatrix}
		2 & -1 \\
		-1 & 2 & -1 \\
		0 & -1 & 2 & -1 \\
		 & & & & \ddots \\ \\ \\
	\end{bmatrix}
	\begin{bmatrix}
		u_1 \\ u_2 \\ \vdots \\ u_m
	\end{bmatrix}
	=
	\begin{bmatrix}
		f(x_1) \\ \\ \vdots \\ f(x_m)
	\end{bmatrix}
\]